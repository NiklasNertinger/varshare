
\documentclass{article} %

%
%
%

%
%

%
%
%

%
%

%
     \usepackage[final,nonatbib]{neurips_2020}

\usepackage[numbers]{natbib}
\usepackage{times}
\usepackage{graphicx, mathtools, amsthm,wrapfig}
\usepackage{thm-restate}
\usepackage[noend]{algorithmic}
\usepackage{amssymb}
\usepackage{thmtools}
\usepackage{mathtools}
\usepackage{amsmath}
%
\usepackage[shortlabels]{enumitem}
\usepackage{sidecap}
\usepackage[capbesideposition=outside,capbesidesep=quad]{floatrow}
\usepackage{float}


%
\input{math_commands.tex}
%% editing comment
\newcommand{\cmt}[1]{{\footnotesize\textcolor{red}{#1}}}
\newcommand{\note}[1]{\cmt{Note: #1}}
%\newcommand{\todo}[1]{\cmt{To-Do: #1}}


%% abbreviations
\newcommand{\x}{\mathbf{x}}
\newcommand{\z}{\mathbf{z}}
\newcommand{\y}{\mathbf{y}}
\newcommand{\w}{\mathbf{w}}
\newcommand{\data}{\mathcal{D}}

\newcommand{\etal}{{et~al.}\ }
\newcommand{\eg}{e.g.\ }
\newcommand{\ie}{i.e.\ }
\newcommand{\nth}{\text{th}}
\newcommand{\pr}{^\prime}
\newcommand{\tr}{^\mathrm{T}}
\newcommand{\inv}{^{-1}}
\newcommand{\pinv}{^{\dagger}}
\newcommand{\real}{\mathbb{R}}
\newcommand{\gauss}{\mathcal{N}}
\newcommand{\norm}[1]{\left|#1\right|}
\newcommand{\trace}{\text{tr}}

%% specifics for the paper
\newcommand{\reward}{r}
\newcommand{\policy}{\pi}
\newcommand{\mdp}{\mathcal{M}}
\newcommand{\states}{\mathcal{S}}
\newcommand{\actions}{\mathcal{A}}
\newcommand{\observations}{\mathcal{O}}
\newcommand{\transitions}{\mathcal{T}}
\newcommand{\initial}{\mathcal{I}}
\newcommand{\horizon}{H}
\newcommand{\rewardevent}{R}
\newcommand{\probr}{p_\rewardevent}
\newcommand{\metareward}{\bar{\reward}}

\newcommand{\pihi}{\pi^{\text{hi}}}
\newcommand{\pilo}{\pi^{\text{lo}}}
\newcommand{\ah}{\mathbf{w}}

\newcommand{\loss}{\mathcal{L}}
\newcommand{\eye}{\mathbf{I}}

\newcommand{\model}{\hat{p}}

\newcommand{\pimix}{\pi_{\text{mix}}}

\newcommand{\pib}{\bar{\pi}}
\newcommand{\epspi}{\epsilon_{\pi}}
\newcommand{\epsmodel}{\epsilon_{m}}

\newcommand{\upd}{\mathcal{U}}

\newcommand{\ind}{\delta}

%% math
\newcommand{\cY}{\mathcal{Y}}
\newcommand{\cX}{\mathcal{X}}
\newcommand{\en}{\mathcal{E}}
\newcommand{\be}{\mathbf{e}}
\newcommand{\by}{\mathbf{y}}
\newcommand{\bx}{\mathbf{x}}
\newcommand{\bz}{\mathbf{z}}
\newcommand{\bo}{\mathbf{o}}
\newcommand{\bs}{\mathbf{s}}
\newcommand{\ba}{\mathbf{a}}
\newcommand{\ot}{\bo_t}
\newcommand{\st}{\bs_t}
\newcommand{\at}{\ba_t}
\newcommand{\op}{\mathcal{O}}
\newcommand{\opt}{\op_t}
\newcommand{\kl}{D_\text{KL}}
\newcommand{\tv}{D_\text{TV}}
\newcommand{\ent}{\mathcal{H}}

\newcommand{\bzhi}{\bz^\text{hi}}

\usepackage{hyperref}
\usepackage{cleveref}
\usepackage{url}

%
\newcommand{\theHalgorithm}{\arabic{algorithm}}

\usepackage{booktabs}       %
\usepackage{amsfonts}       %
\usepackage{nicefrac}       %
\usepackage{microtype, todonotes}      %
\usepackage{algorithm}
\usepackage{algorithmic}
%
\usepackage{subcaption}
\usepackage{enumitem}

\usepackage{amsthm, amssymb, bbm, bm}
\usepackage{amsmath}
\usepackage{fleqn, tabularx}
\usepackage{multirow}
\usepackage[export]{adjustbox}
\usepackage{titlesec}
\usepackage{lipsum}

\hypersetup{
 colorlinks=True,
 linkcolor=blue,
 citecolor=blue,
 urlcolor=blue}


\makeatletter
\newcommand{\mybox}{%
    \collectbox{%
        \setlength{\fboxsep}{1pt}%
        \fbox{\BOXCONTENT}%
    }%
}
\makeatother



\newcommand{\arxiv}[1]{\textcolor{black}{#1}}
\newcommand{\icml}[1]{\textcolor{black}{#1}}
\newcommand{\KH}[1]{\textcolor{black}{#1}}
\newcommand{\icmllast}[1]{\textcolor{black}{#1}}
\newcommand{\neurips}[1]{\textcolor{black}{#1}}

\newcommand{\fix}{\marginpar{FIX}}
\newcommand{\new}{\marginpar{NEW}}

\title{Gradient Surgery for Multi-Task Learning}

\author{%
 Tianhe Yu$^1$, Saurabh Kumar$^1$, Abhishek Gupta$^2$, Sergey Levine$^2$,\\
 \textbf{Karol Hausman}$^3$, \textbf{Chelsea Finn}$^1$\\
 Stanford University$^1$, UC Berkeley$^2$, Robotics at Google$^3$\\
 \texttt{tianheyu@cs.stanford.edu} \\
 %
 %
 %
 %
 %
 %
 %
 %
 %
 %
 %
 %
 %
 %
 %
 %
 %
 %
 %
 %
 %
}

\begin{document}

\maketitle

\titlespacing\section{0pt}{2pt plus 2pt minus 2pt}{0pt plus 2pt minus 2pt}
\titlespacing\subsection{0pt}{2pt plus 3pt minus 2pt}{0pt plus 2pt minus 2pt}
\setlength{\textfloatsep}{5pt plus 2.0pt minus 1.0pt}
\setlength{\dbltextfloatsep}{7pt plus 2.0pt minus 1.0pt}
%


\begin{abstract}
While deep learning and deep reinforcement learning (RL) systems have demonstrated impressive results in domains such as image classification, game playing, and robotic control, data efficiency remains a major challenge. %
Multi-task learning has emerged as a promising approach for sharing structure across multiple tasks to enable more efficient learning. However, the multi-task setting presents a number of optimization challenges, making it difficult to realize large efficiency gains compared to learning tasks independently.
The reasons why multi-task learning is so challenging compared to single-task learning are not fully understood.
\icml{In this work, we identify a set of three conditions of the multi-task optimization landscape that cause detrimental gradient interference,} and develop a simple yet general approach for avoiding such interference between task gradients. We propose a form of gradient surgery that projects a task's gradient onto the normal plane of the gradient of any other task that has a \emph{conflicting} gradient.
On a series of challenging multi-task supervised and multi-task RL problems,  this approach leads to substantial gains in efficiency and performance. Further, it is model-agnostic and can be combined with previously-proposed multi-task architectures for enhanced performance. 
\end{abstract}

\section{Introduction}

While deep learning and deep reinforcement learning (RL) have shown considerable promise in enabling systems to learn complex tasks, the data requirements of current methods make it difficult to learn a breadth of capabilities, particularly when all tasks are learned individually from scratch. A natural approach to such multi-task learning problems is to train a network on all tasks jointly, with the aim of discovering shared structure across the tasks in a way that achieves greater efficiency and performance than solving tasks individually. However, learning multiple tasks all at once results is a difficult optimization problem, sometimes leading to \emph{worse} overall performance and data efficiency compared to learning tasks individually~\citep{parisotto2015actor,rusu2015policydistillation}. These optimization challenges are so prevalent that multiple multi-task RL algorithms have considered using independent training as a subroutine of the algorithm before distilling the independent models into a multi-tasking model~\citep{levine2016end,parisotto2015actor,rusu2015policydistillation,ghosh2017dnc,teh2017distral}, producing a multi-task model but losing out on the efficiency gains over independent training. If we could tackle the optimization challenges of multi-task learning effectively, we may be able to actually realize the hypothesized benefits of multi-task learning without the cost in final performance.

\begin{wrapfigure}{R}{0.59\textwidth}
%
    \centering
    \includegraphics[width=1.0\linewidth]{figures/pcgrad_example.png}
    \vspace{-0.4cm}
    \caption{\footnotesize Visualization of \arxiv{PCGrad} on a 2D multi-task optimization problem. (a) A multi-task objective landscape. (b) \& (c) Contour plots of the individual task objectives that comprise (a). (d) Trajectory of gradient updates on the multi-task objective using the Adam optimizer. The gradient vectors of the two tasks at the end of the trajectory are indicated by blue and red arrows, where the relative lengths are on a log scale.(e) Trajectory of gradient updates on the multi-task objective using Adam with PCGrad. For (d) and (e), the optimization trajectory goes from black to yellow.}
    \vspace{-0.3cm}
    \label{fig:optlandscape}
%
\end{wrapfigure}

While there has been a significant amount of research in multi-task learning~\citep{caruana97multitask,ruder2017overview}, the optimization challenges are not well understood. Prior work has described varying learning speeds of different tasks~\citep{chen2017gradnorm,hessel2019popart} and \arxiv{plateaus} in the optimization landscape~\citep{schaul2019ray} as potential causes, whereas a range of other works have focused on the model architecture~\citep{misra2016crossstitch,liu2018attention}. In this work, we instead hypothesize that one of the main optimization issues in multi-task learning arises from gradients from different tasks conflicting with one another in a way that is detrimental to making progress. We define two gradients to be conflicting if they point away from one another, i.e., have a negative cosine similarity. We hypothesize that such conflict is detrimental when a) conflicting gradients coincide with b) high positive curvature and c) a large difference in gradient magnitudes.

As an illustrative example, consider the 2D optimization landscapes of two task objectives in Figure~\ref{fig:optlandscape}a-c.
\icml{The optimization landscape of each task consists of a deep valley, a property that has been observed in neural network optimization landscapes~\citep{goodfellow2014qualitatively}, and the bottom of each valley is characterized by high positive curvature and large differences in the task gradient magnitudes.
%
Under such circumstances, the multi-task gradient is dominated by one task gradient, which comes at the cost of degrading the performance of the other task. Further, due to high curvature, the improvement in the dominating task may be overestimated, while the degradation in performance of the non-dominating task may be underestimated.
As a result, the optimizer struggles to make progress on the optimization objective. In Figure~\ref{fig:optlandscape}d), the optimizer reaches the deep valley of task 1, but is unable to traverse the valley in a parameter setting where there are \icmllast{\emph{conflicting gradients}, \emph{high curvature}, and \emph{a large difference in gradient magnitudes}} (see gradients plotted in Fig.~\ref{fig:optlandscape}d).
In Section~\ref{sec:analysis}, we find experimentally that this \emph{tragic triad}
also occurs in a higher-dimensional neural network multi-task learning problem.}

The core contribution of this work is a method for mitigating gradient interference by altering the gradients directly, i.e. by performing ``gradient surgery.'' If two gradients are conflicting, we alter the gradients by projecting each onto the normal plane of the other, preventing the interfering components of the gradient from being applied to the network. We refer to this particular form of gradient surgery as \emph{projecting conflicting gradients} (PCGrad). PCGrad is model-agnostic, requiring only a single modification to the application of gradients. Hence, it is easy to apply to a range of problem settings, including multi-task supervised learning
and multi-task reinforcement learning, and can also be readily combined with other multi-task learning approaches, such as those that modify the architecture. \icml{We theoretically prove the local conditions under which PCGrad improves upon standard multi-task gradient descent}, and we empirically evaluate PCGrad on a variety of challenging problems, including multi-task CIFAR classification, multi-objective scene understanding, a challenging multi-task RL domain, and goal-conditioned RL. Across the board, we find PCGrad leads to substantial improvements in terms of data efficiency, optimization speed, and final performance compared to prior approaches, including a more than 30\% absolute improvement in multi-task reinforcement  learning problems. Further, on multi-task supervised learning tasks, PCGrad can be successfully combined with prior state-of-the-art methods for multi-task learning for even greater performance. 




%



\newcommand{\1}{\mathbb{I}}

\section{Meta-World}
\vspace{-0.2cm}

%%CF: Would love for someone to take a pass on the below and give suggestions / make edits. Not sure if it's too formulaic or verbose. Though, I would also like to get across that we put careful consideration of many design decisions into the benchmark.

If we want meta-RL methods to generalize effectively to entirely new tasks, we must meta-train on broad task distributions that are representative of the range of tasks that a particular agent might need to solve in the future. %Such a suite of diverse tasks would exercise the full potential of both meta-RL and multi-task RL methods, and must consist of a large number of tasks spanning both parametric and non-parametric degrees of variation.
To this end, we propose a new multi-task and meta-RL benchmark, which we call Meta-World. In this section, we motivate the design decisions behind the Meta-World tasks, discuss
%discuss %the design decisions behind Meta-World,
the range of tasks, describe the representation of the actions, observations, and rewards, and present a set of evaluation protocols of varying difficulty for both meta-RL and multi-task RL.

%\subsection{Design Considerations for a Multi-Task and Meta-RL Benchmark}

%We now describe the Meta-World benchmark, with which we aim to provide both a widely accessible suite of tasks and a standard evaluation protocol for meta-learning and multi-task learning RL methods.
%An effective benchmark must (1) reflect important, realistic, and unsolved problems, such that progress on the benchmark leads to meaningful progress on real problems; (2) be at the right level of difficulty: challenging for existing methods, yet not too difficult as to make research progress impossible; (3) be widely accessible and easy to use.
%We design MetaWorld with each of these considerations in mind.
%%SL: commented out above sentence since it should be obvious (just trying to cut down on space a bit)
%The primary unsolved problem that we focus on for is the meta-RL problem: learning models that can quickly acquire new tasks by leveraging experience from other, related prior tasks. Since this problem is unsolved, we cannot ensure (2) fully. However, we can provide versions of our benchmark at variable levels of difficulty, such that easier variants demand less broad generalization. By ensuring that easier versions of our benchmark can be solved with current methods, we provide a tiered evaluation protocol that will enable progress of meta-RL and multi-task RL methods.

%For (1), the particular real problem we focus on primarily is the meta-RL problem of quickly learning distinctly new tasks by leveraging experience on other, previously seen tasks. Since this problem is inherently unsolved, it is impossible to ensure (2). However, there are a number of steps that will make the benchmark well-suited for future research. This includes: making versions of the evaluation at variable levels of difficulty, ensuring that the easiest versions of the benchmark can be solved by current methods and, most importantly, taking careful consideration of the distribution of tasks to ensure both diversity and shared structure.
%%Without both of these, generalization to new tasks is not possible.
%%SL: likewise I think that last sentence can be cut? but feel free to uncomment if you think it's critical

%%SL.7.6: moved this to the top
%In the remainder of this section, we will discuss each of these challenges and design decisions in greater detail, share the insights we learned when designing task distributions that are well suited for algorithms to make progress on, and describe the evaluation protocol, both for meta-learning algorithms as well as for multi-task learning methods.


%Since generalization to new tasks requires a sufficient diversity of meta-training tasks, MetaWorld consists of $50$ qualitatively distinct manipulation task families.
%%SL.7.4: can we reference a figure here?
%Each of these task families itself consists of an infinite number of tasks, which can be generated by varying the start and goal positions of objects in the scene.
%For each task, different variations such as goal positions are also implemented, which are used to evaluate if meta-RL algorithms can generalize to new variation within a control task, and also to facilitate the potential for greater overlap across distinct tasks.
%Below, we discuss the design of the tasks, the parametric variability in each task family, the reward functions and binary success metrics for evaluation, and the proposed evaluation protocols.













%MetaWorld allows us to test if multi-task RL approaches can learn a single policy that can solve multiple tasks, and whether meta-RL algorithms can generalize to new tasks that were not seen during meta-training.

\subsection{The Space of Manipulation Tasks: Parametric and Non-Parametric Variability}
\label{sec:parametric}

\begin{wrapfigure}{r}{0.5\textwidth}
    \centering
    \includegraphics[width=0.5\columnwidth]{figures_v2/variation_cartoon.pdf}
    \caption{\footnotesize Parametric/non-parametric variation: all ``reach puck'' tasks (left) can be parameterized by the puck position, while the difference between ``reach puck'' and ``open window'' (right) is non-parametric.}
    \vspace{-0.5cm}
    \label{fig:variation-catoon}
\end{wrapfigure}

%\cf{This section should motivate our design choices about talking about the importance of parametric and non-parametric variation, without being too repetitive with the problem statements above. Hence, if we hope to scale these algorithms to more realistic domains, such as the domain of table top manipulation tasks, we requires a suite of tasks that reflects the breadth of tasks in the domain while structured in a way that is still amenable to transfer (albeit a more difficult form of transfer than previous evaluations). However, with enough tasks, even two qualitatively different tasks may overlap, serving as a catalyst for generalization. For example, consider a robot that needs to push a block away from itself versus push a drawer closed. For some initial positions of the object and the drawer, these tasks are the nearly the same, \emph{as long as the reward functions is structured in a similar way}.}

A task, $\task$, in Meta-World is defined as the tuple \textit{(reward\; function, \;initial \;object \;position, \;target \;position)}
Meta-learning makes two critical assumptions: first, that the meta-training and meta-test tasks are drawn from the same distribution, $p(\task)$, and second, that the task distribution $p(\task)$ exhibits shared structure that can be utilized for efficient adaptation to new tasks. 
If $p(\task)$ is defined as a family of variations within a particular control task, as in prior work~\cite{finn2017model,rakelly2019efficient}, then it is unreasonable to hope for generalization to entirely new control tasks. For example, an agent has little hope of being able to quickly learn to open a door, without having ever experienced doors before, if it has only been trained on a set of meta-training tasks that are homogeneous and narrow. 
Thus, to enable meta-RL methods to adapt to entirely new tasks, we propose a much larger suite of tasks consisting of $50$ qualitatively-distinct manipulation tasks, where continuous parameter variation cannot be used to describe the differences between tasks.

%Thus, instead, the tasks in the task distribution must be

%structurally similar for transfer to be successful, yet sufficiently diverse to achieve generalization to distinctly new tasks.

%Therefore, learning such structure is regarded as the main goal of meta-learning. In prior meta-RL work, $p(\task)$ is defined as a family of variations within a particular control task~\cite{finn2017model, rakelly2019efficient}, such as a distribution of goal velocities for a running robot. How can we hope to generalize to entirely new tasks -- e.g., generalize such that the agent can quickly learn to open a door without having ever experienced doors before -- when the set of meta-training tasks is so homogeneous and narrow?




% Each of the manipulation tasks in our benchmark defines a full task family, where individual tasks can be obtained by instantiating the task-relevant objects (e.g., the door in the door opening task) at a randomized location and orientation, and with a randomized goal. The range of potential start and goal positions and orientations for each object is defined for each task. 
%If the locations are not fixed, this kind of memorization is impossible, and the model is forced to generalize more broadly. Note that this kind of parametric variation, which we introduce \emph{for each task family}, essentially represents the entirety of the task distribution for previous meta-RL evaluations~\cite{finn2017model, rakelly2019efficient}, which test on single task families (e.g., running in a direction) with parametric variability (e.g., variation in the goal direction). Our full task distribution is therefore substantially broader, since it includes this parametric variability \emph{for each of the $50$ task families}. In order to evaluate multi-task RL methods, we also provide a parametrically static version of the tasks, where the object and goal configurations are fixed within each task.


With such non-parametric variation, however, there is the danger that tasks will not exhibit enough shared structure, or will lack the task overlap needed for the method to avoid memorizing each of the tasks. 
Motivated by this challenge, we design each task to include parametric variation in object and goal positions, as illustrated in Figure~\ref{fig:variation-catoon}.
Introducing this parametric variability not only creates a substantially larger (infinite) variety of tasks, but also makes it substantially more practical to expect that a meta-trained model will generalize to acquire entirely new tasks more quickly, since varying the positions provides for wider coverage of the space of possible manipulation tasks. Without parametric variation, the model could for example memorize that any object at a particular location is a door, while any object at another location is a drawer. 
If the locations are not fixed, this kind of memorization is much less likely, and the model is forced to generalize more broadly. 
With enough tasks and variation within tasks, pairs of qualitatively-distinct tasks are more likely to overlap, serving as a catalyst for generalization.
For example, closing a drawer and pushing a block can appear as nearly the same task for some initial and goal positions of each object.


Note that this kind of parametric variation, which we introduce \emph{for each task}, essentially represents the entirety of the task distribution for previous meta-RL evaluations~\cite{finn2017model, rakelly2019efficient}, which test on single tasks (e.g., running towards a goal) with parametric variability (e.g., variation in the goal position). Our full task distribution is therefore substantially broader, since it includes this parametric variability \emph{for each of the $50$ tasks}.


To provide shared structure, the $50$ environments require the same robotic arm to interact with different objects, with different shapes, joints, and connectivity. The tasks themselves require the robot to execute a combination of reaching, pushing, and grasping, depending on the task.
By recombining these basic behavioral building blocks with a variety of objects with different shapes and articulation properties, we can create a wide range of manipulation tasks. For example, the \textbf{open door} task
%%SL.7.4: reference a figure
involves pushing or grasping an object with a revolute joint, while the \textbf{open drawer} task requires pushing or grasping an object with a sliding joint.
%%SL.7.4: reference a figure
More complex tasks require a combination of these building blocks, which must be executed in the right order.
We visualize all of the tasks in Meta-World in Figure~\ref{fig:ml45_teaser}, and include a description of all tasks in Appendix~\ref{app:tasks}.
%We list some examples of these more complex tasks below, provide visualizations of all the tasks in Figure~\ref{fig:teaser}, and include a description of all tasks in Appendix~\ref{app:tasks}.

%%SL.7.6: My suggestion would be to cut the examples below -- I think a picture is worth a thousand words here, and the reader will get a much clearer idea of what we are actually doing by looking at Figure 1 (which should be made, and should be good). We're going to run out of space, so in terms of thinking about what we can cut, I feel like these three examples are the most cuttable
\iffalse
\textbf{Button pressing:} The robot needs to press a button located on the wall. This task is a composition of reaching and pushing primitives, as the robot is required to first reach the button and then push it all the way to the end.
%%SL.7.4: I'm not sure this is the right task to highlight -- we mentioned before that we are going to talk about more complex tasks, but this task seems to be quite trivial. Also, reference a figure.
\todo{Kevin / Deirdre -- please pick a different example for this one.}

\textbf{Shelf placing:} The robot needs to place an object onto a 2-level shelf. Reaching the object, grasping the object, and placing the object on the shelf should be performed sequentially for this task.
%%SL.7.4: Reference a figure.

\textbf{Stick pushing:} The robot needs to pick up a stick and use it to push a box to a goal position. In this task, the sequence of primitives to be performed is reaching the stick, grasping the stick, placing the stick near the box, and pushing the box.
%%SL.7.4: Reference a figure.
\fi

All of the tasks are implemented in the MuJoCo physics engine~\cite{todorov2012mujoco}, which enables fast simulation of physical contact. 
To make the interface simple and accessible, we base our suite on the Multiworld interface~\cite{nair2018visual} and the OpenAI Gym environment interfaces~\cite{brockman2016openai}, making additions and adaptations of the suite relatively easy for researchers already familiar with Gym.


\iffalse
%%CF.7.5: I'm commenting this out, since it is covered in the previous section now.
\subsection{Parametric Variability}
\label{sec:parametric}

To encourage easier generalization to new tasks,
%%SL.7.4: Let's try to stay away from the word "encourage" -- it's very informal and imprecise, and doesn't really sound very flattering in an optimization context.
in addition to designing a large number of heterogeneous tasks, having a heavily overlapped observation space among all tasks is conducive.
%%SL.7.4: This seems like a pretty roundabout way of putting it. Maybe say something like: In order to be able to train one model on all of the tasks, we must define a state representation that is compatible with all of the task domains.
The rationale is that the agent is more inclined to discover the underlying shared structure among tasks if it interacts with a variety of environments that enjoy similar but varied representations. We enable such representations of the observations by drawing positions of objects and goals randomly from a common set across tasks and setting them parametrically.
%%SL.7.4: I'm confused, I thought you were talking about representations (i.e., how the state vector is put together), but now suddenly this is talking about positions of objects and goals? Overall, I don't think this paragraph really makes much sense the way it is written. Maybe a potential rephrasing could go like this:
% Each of the manipulation tasks in our benchmark defines a full task family, where individual tasks can be obtained by instantiating the task-relevant objects (e.g., the door in the door opening task) at a randomized location and orientation, and with a randomized goal. The range of potential start and goal positions and orientations for each object is defined for each task. Introducing this parametric variability not only creates a substantially larger (infinite) variety of tasks, but also makes it substantially more practical to expect that a meta-trained model will generalize to acquire entirely new tasks more quickly, since varying the positions provides for wider coverage of the space of possible manipulation tasks. Without parametric variation, the model could for example memorize that any object at a particular location is a door, while any object at another location is a drawer. If the locations are not fixed, this kind of memorization is impossible, and the model is forced to generalize more broadly. Note that this kind of parametric variation, which we introduce \emph{for each task family}, essentially represents the entirety of the task distribution for previous meta-RL evaluations~\cite{finn2017model, rakelly2019efficient}, which test on single task families (e.g., running in a direction) with parametric variability (e.g., variation in the goal direction). Our full task distribution is therefore substantially broader, since it includes this parametric variability \emph{for each of the $50$ task families}. In order to evaluate multi-task RL methods, we also provide a parametrically static version of the tasks, where the object and goal configurations are fixed within each task.
Therefore, the robot is expected to experience similar but a wide range of object and goal configurations across all tasks.

Note that the distribution of such a parametric variability is essentially the task distribution used for evaluating previous meta-RL algorithms ~\cite{finn2017model, rakelly2019efficient}. Hence our task distribution strictly covers the distributions used in prior work and poses a more challenging adaptation problem for meta-RL algorithms.

In order to evaluate multi-task RL methods, we also provide a parametrically static version of the tasks, where the object and goal configurations are fixed within each task.
\fi

\subsection{Actions, Observations, and Rewards}



%\todo{Need to describe the observation space here, whether or not it includes goal information, how we make it fixed in length, etc. Can motivate using low-dim state space by saying that we don't want it to be too hard (yet), similar to the motivation in the next paragraph}

In order to represent policies for multiple tasks with one model, the observation and action spaces must contain significant shared structure across tasks. All of our tasks are performed by a simulated Sawyer robot. The action space is a 2-tuple consisting of the change in 3D space of the end-effector followed by a normalized torque that the gripper fingers should apply. The actions in this space range between $-1$ and $1$.
For all tasks, the robot must either manipulate one object with a variable goal position, or manipulate two objects with a fixed goal position. The observation space is represented as a 6-tuple of the 3D Cartesian positions of the end-effector, a normalized measurement of how open the gripper is, the 3D position of the first object, the quaternion of the first object, the 3D position of the second object, the quaternion of the second object, all of the previous measurements in the environment, and finally the 3D position of the goal. If there is no second object or the goal is not meant to be included in the observation, then the quantities corresponding to them are zeroed out. The observation space is always $39$ dimensional.
%A more general observation space would be to use image pixels; we leave this for future development, however, because RL with raw images presents a major challenge for existing RL methods, even without multiple tasks.

%While meta-RL algorithms can learn from this plain observation space, multi-task RL algorithms need to use a observation space that is augmented with task identities to retrieve task context information. We provide this augmented observation space by concatenating a one-hot vector, which represents a task identity, with the original observation.
%%CF: I commented out the above, since we now discuss it in section 5.

%\cf{This section should motivate by discussing the importance of having shared structure across the reward functions, along with designing tasks that are solveable.}

Designing reward functions for Meta-World requires two major considerations. First, to guarantee that our tasks are within the reach of current single-task reinforcement learning algorithms, which is a prerequisite for evaluating multi-task and meta-RL algorithms, 
%%CF.7.5: Arguably it's not a pre-requisite, since the other tasks should make learning easier. Can we rephrase? (still emphasizing that making the tasks easy enough is important)
we design well-shaped reward functions for each task that make each of the tasks at least individually solvable.
%%CF.7.5: Also mention that sparse reward settings is a straightforward extension of the benchmark.
More importantly, the reward functions must exhibit shared structure across tasks. Critically, even if the reward function admits the same optimal policy for multiple tasks, varying reward scales or structures can make the tasks appear completely distinct for the learning algorithm, masking their shared structure and leading to preferences for tasks with high-magnitude rewards~\cite{DBLP:journals/corr/abs-1809-04474}. 
%should also follow similar structure and have the same magnitudes across all tasks. Otherwise, meta- and multi-task- RL algorithms would favor the tasks that provider higher-magnitude rewards, while ignoring other tasks with lower reward magnitude.
Accordingly, we adopt a structured, multi-component
reward function for all tasks, which leads to effective policy learning for each of the task components. For instance, in a task that involves a combination of reaching, grasping, and placing an object, let $o \in \mathbb{R}^3$ be the object position, where $o = (o_x, o_y, o_z)$, $h \in \mathbb{R}^3$ be the position of the robot's gripper, $z_\text{target} \in \mathbb{R}$ be the target height of lifting the object, and $g \in \mathbb{R}^3$ be goal position. With the above definition, the multi-component reward function $R$ is the combination of a reaching reward, a grasping reward, and a placing reward or subsets thereof for simpler tasks that only involve reaching and/or pushing.
With this design, the reward functions across all tasks have a similar magnitude that ranges between $0$ and $10$, where $10$ always corresponds to the reward-function being solved, and conform to similar structure, as desired. The full form of the reward function and a list of all task rewards is provided in Appendix~\ref{app:rewardfns}.



\iffalse
\begin{figure}[!t]
    \centering
    \includegraphics[width=\columnwidth]{figures/evaluation_table.png}
    \vspace{-0.6cm}
    \caption{List of environments used for evaluating multi-task and meta-RL algorithms.
    \chelsea{See comments.}
    }
    \vspace{-0.4cm}
    \label{fig:evaluation}
\end{figure}
\fi
%%SL.7.4: I'm not sure this is the best way to present the benchmarks. Maybe we can provide a full table listing all the tasks, a bit of information for each one, and for each one an indication of whether it is training/test and for which variant of the benchmark? Also, make sure to have an informative and clear caption.
%%CF: I removed this, since we can illustrate train/test splits for hard mode in Figure 1. I also described the train/test splits for the remaining modes in the text. I think it could still be good to have a very abbreviated version of this table that illustrates the protocols, but does not list out the tasks. We should also be sure to have a description of each of the tasks in the appendix.

\iffalse
\begin{table}[t]
    \centering
    \footnotesize
    \begin{tabular}{c|c|c}
\hline
\footnotesize Task & Description & Roles \\
\hline
reach & reach position & MT, 1, 2tr, 3tr\\
push & & MT, 1, 2tr, 3tr\\
pick and place & & MT, 1, 2tr, 3tr\\
door open & & MT, 2tr, 3tr\\
\hline
\end{tabular}
\quad
\begin{tabular}{c|c|c}
\hline
Task & Description & Roles \\
\hline
bin pick &  &  3ts\\
box open &  &  3ts\\
box close &  &  3ts\\
door lock &  &  3ts\\
door unlock &  &  3ts\\
\hline
\end{tabular}
    \caption{CF: Here's a starter for what an alternative table might look like}
    \label{tab:my_label}
\end{table}
\fi

%%SL.6.29: A few high-level comments here:
% 3. It's important to motivate *why* we chose to do it this way, rather than throw out the protocol without any justification.
% 4. It's also important to be concrete and specific, maybe we add a table for each mode to list the particular tasks in each one?
\subsection{Evaluation Protocol}
\begin{figure}
    \centering
    \includegraphics[width=\columnwidth]{figures_v2/figure_3_metaworld.pdf}
    \caption{\footnotesize Visualization of three of our multi-task and meta-learning evaluation protocols, ranging from within task adaptation in ML1, to multi-task training across 10 distinct task families in MT10, to adapting to new tasks in ML10. Our most challenging evaluation mode ML45 is shown in Figure~\ref{fig:ml45_teaser}.
    }
    \label{fig:evaluation}
\end{figure}
With the goal of providing a challenging benchmark to facilitate progress in multi-task RL and meta-RL, we design an evaluation protocol with varying levels of difficulty, ranging from the level of current goal-centric meta-RL benchmarks to a setting where methods must learn distinctly new, challenging manipulation tasks based on diverse experience across 45 tasks. We hence divide our evaluation into five categories, which we describe next. We then detail our evaluation criteria.


%Tasks used for each category are presented in Figure~\ref{fig:evaluation}.

%%SL.7.4: While I appreciate not having "baby" in there, perhaps it would be best for us to name these evaluations by some concise one or two word name that can be used to refer to them, that is somewhat more informative than just a number? E.g., "single-family, small-scale, large-scale?"
\textbf{Meta-Learning 1 (ML1): Few-shot adaptation to goal variation within one task.} The simplest evaluation aims to verify that previous meta-RL algorithms can adapt to new object or goal configurations on only one type of task. ML1 uses single Meta-World Tasks, with the meta-training ``tasks'' corresponding to $50$ random initial object and goal positions, and meta-testing on $50$ held-out positions. This resembles the evaluations in prior works~\cite{finn2017model, rakelly2019efficient}. We evaluate algorithms on three individual tasks from Meta-World: reaching, pushing, and pick and place, where the variation is over reaching position or goal object position. The goal positions are not provided in the observation, forcing meta-RL algorithms to adapt to the goal through trial-and-error.

\textbf{Multi-Task 1 (MT1): Learning one multi-task policy that generalizes to 50 tasks belonging to the same environment}. This evaluation aims to verify how well multi-task algorithms can learn across a large related task distribution. MT1 uses single Meta-World environments, with the training ``tasks'' corresponding to $50$ random initial object and goal positions. The goal positions are provided in the observation and are a fixed set, as to focus on the ability of algorithms in acquiring a distinct skill across multiple goals, rather than generalization and robustness.

\textbf{Multi-Task 10, Multi-Task 50 (MT10, MT50): Learning one multi-task policy that generalizes to 50 tasks belonging to 10 and 50 training environments, for a total of 500, and 2,500 training tasks.} A first step towards adapting quickly to distinctly new tasks is the ability to train a single policy that can solve multiple distinct training tasks. The multi-task evaluation in Meta-World tests the ability to learn multiple tasks at once, without accounting for generalization to new tasks. The MT10 evaluation uses 10 environments: reach, push, pick and place, open door, open drawer, close drawer, press button top-down, insert peg side, open window, and open box. The larger MT50 evaluation uses all $50$ Meta-World environments. In our experiments, the algorithm is typically provided with a one-hot vector indicating the current task. The positions of objects and goal positions are fixed in all tasks in this evaluation, so as to focus on acquiring the distinct skills, rather than generalization and robustness.

%A first step towards adapting quickly to distinctly new tasks is the ability to train a single policy that can solve multiple distinct training tasks.
%We develop an multi-task evaluation, MT10, on 10 tasks: reach, push, pick and place, open door, open drawer, close drawer, pess button top-down, insert peg side, open window, and open box, as well as en evaluation on all $50$ MetaWorld tasks (MT50).
%listed in Figure~\ref{fig:evaluation}.
%%SL.6.29: which ones?
% Since there do not need to evaluate generalization to new tasks,
%%SL.6.29: This is phrased in contrast to the meta-learning section below, which does require generalization, but it comes *before* the meta-learning section, so the reader won't understand the significance of this.
%Since multi-task RL algorithms require task identifications as input to the policy, w
%We identify each task with a one-hot encoding, which is provided with the current state information to multi-task policy. As we are evaluating the ability to acquire 10 qualitatively different tasks, we fix the position of the objects and goal positions in each task to ease multi-task RL training and encourage the RL agent to focus on learning different skills. 


\textbf{Meta-Learning 10, Meta-Learning 45 (ML10, ML45): Few-shot adaptation to new test tasks with 10 and 50 meta-training tasks.}  With the objective to test generalization to new tasks, we hold out 5 tasks and meta-train policies on 10 and 45 tasks.
We randomize object and goals positions %, as per Section~\ref{sec:parametric}, 
and intentionally select training tasks with structural similarity to the test tasks. Task IDs are not provided as input, requiring a meta-RL algorithm to identify the tasks from experience.
%Similar to ML9, object and goal positions are varied and task IDs are not provided in the observation space.

%%SL.7.4: Overall, I'm pretty concerned that the current description of the meta-learning protocol is very hard to understand. Maybe let me know once you've had a chance to revise, and I'll take another quick look (and/or edit it directly)?
%%CF: I took another pass.

\textbf{Success metrics.} 
Since values of reward are not directly indicative how successful a policy is, we define an interpretable success metric for each task, which will be used as the evaluation criterion for all of the above evaluation settings.
Since all of our tasks involve manipulating one or more objects into a goal configuration, this success metric is typically based on the distance between the task-relevant object and its final goal pose, i.e. $\|o - g\|_2 < \epsilon$, where $\epsilon$ is a small distance threshold such as $5$ cm. For the complete list of success metrics and thresholds for each task, see Appendix~\ref{tbl:task_metrics}.
%%SL.7.4: Is it 5 cm everywhere? If not, reference an appendix and give a table of thresholds in the appendix. Being precise is important when describing an open-source benchmark.





%\subsection{Multi-Task Evaluation Protocol}
%%SL.7.4: I would put this last (or just delete it -- at this point I'm really starting to get the impression that keeping the multi-task stuff around harms more than it helps).

%%KH.7.2 Can we try to keep the equal focus on both, multi-task RL and meta-RL? Maybe we can avoid saying that the main goal is to study meta-learning. We should also write here the evaluation protocol for multi-task RL (preferably written in a similar form as meta-RL).
%%KH.7.2 Should we also evaluate multi-task RL in the hard mode?



\section{Related Work}

Algorithms for multi-task learning typically consider how to train a single model that can solve a variety of different tasks~\citep{caruana97multitask, bakker2003clustering, ruder2017overview}.
The multi-task formulation has been applied to many different settings, including supervised learning~\citep{zhang2014facial, long2015learning, yang2016trace, sener2018multi, zamir2018taskonomy} and reinforcement-learning~\citep{espeholt2018impala, wilson2007multi}, as well as many different domains, such as vision~\citep{bilen2016integrated, misra2016cross, kokkinos2017ubernet, liu2018attention, zamir2018taskonomy}, language~\citep{collobert2008unified, dong2015multi, mccann2018natural, radford2019language} and robotics~\citep{riedmiller2018learning, wulfmeier2019regularized, hausman2018learning}.
While multi-task learning has the promise of accelerating acquisition of large task repertoires, in practice it presents a challenging optimization problem, which has been tackled in several ways in prior work.

A number of architectural solutions have been proposed to the multi-task learning problem based on multiple modules or paths~\citep{fernando2017pathnet, devin2016modularnet, misra2016crossstitch, rusu2016progressive, rosenbaum2017routing, vandenhende2019branched, rosenbaum2017routing}, or using attention-based architectures~\citep{liu2018attention, maninis2019attention}. Our work is agnostic to the model architecture and can be combined with prior architectural approaches in a complementary fashion.
A different set of multi-task learning approaches aim to decompose the problem into multiple local problems, often corresponding to each task, that are significantly easier to learn, akin to divide and conquer algorithms~\citep{levine2016end,rusu2015policydistillation,parisotto2015actor,teh2017distral, ghosh2017dnc, czarneki2019distilling}. 
Eventually, the local models are combined into a single, multi-task policy using different distillation techniques (outlined in~\citep{hinton2015distilling,czarneki2019distilling}).
In contrast to these methods, we propose a simple and cogent scheme for multi-task learning that allows us to learn the tasks simultaneously using a single, shared model without the need for network distillation.


Similarly to our work, a number of prior approaches have observed the difficulty of optimization in multi-task learning~\citep{hessel2019popart, kendall2018multitask, schaul2019ray, suteu2019regularizing}. Our work suggests that the challenge in multi-task learning may be attributed to what we describe as the tragic triad of multi-task learning (i.e., conflicting gradients, high curvature, and large gradient differences), 
which we address directly by introducing a simple and practical algorithm that deconflicts gradients from different tasks. Prior works combat optimization challenges by rescaling task gradients ~\cite{sener2018multi,chen2018gradnorm}. We alter both the magnitude and direction of the gradient, which we find to be critical for good performance (see Fig.~\ref{fig:rl_results}). 
Prior work has also used the cosine similarity between gradients to define when an auxiliary task might be useful~\citep{du2018aux} or when two tasks are related~\cite{suteu2019regularizing}. We similarly use cosine similarity between gradients to determine if the gradients between a pair of tasks are in conflict. %
Unlike~\citet{du2018aux}, we use this measure for effective multi-task learning, instead of ignoring auxiliary objectives.
Overall, we empirically compare our approach to a number of these prior approaches~\cite{sener2018multi,chen2018gradnorm,suteu2019regularizing}, and observe superior performance with PCGrad.
%

\arxiv{Multiple approaches to continual learning have studied how to prevent gradient updates from adversely affecting previously-learned tasks through various forms of gradient projection\icmllast{~\citep{lopez2017gradient, agem, farajtabar2019orthogonal, guo2020improved}}. These methods focus on sequential learning settings, and solve for the gradient projections using quadratic programming~\citep{lopez2017gradient}, only project onto the normal plane of the average gradient of past tasks~\citep{agem}, or project the current task gradients onto the orthonormal set of previous task gradients~\citep{farajtabar2019orthogonal}. In contrast, our work focuses on positive transfer when simultaneously learning multiple tasks, does not require solving a QP, and \emph{iteratively} projects the gradients of each task instead of \emph{averaging} \icmllast{or only projecting the \emph{current} task gradient.}} Finally, our method is distinct from and solves a different problem than the projected gradient method~\citep{gradient_projection}, which is an approach for constrained optimization that projects gradients onto the constraint manifold. 


\section{Experimental Results and Analysis}
\vspace{-0.2cm}
\label{sec:result}

The first, most basic goal of our experiments is to verify that each of the $50$ presented tasks are indeed solveable by existing single-task reinforcement learning algorithms. We provide this verification in Appendix~\ref{app:singletask}. Beyond verifying the individual tasks, the goals of our experiments are to study the following questions: (1) can existing state-of-the-art meta-learning algorithms quickly learn qualitatively new tasks when meta-trained on a sufficiently broad, yet structured task distribution, and (2) how do different multi-task and meta-learning algorithms compare in this setting? %, and (3) what are the primary challenges that existing algorithms face when presented with the broader task distributions in Meta-World?
%\subsection{Comparative Evaluation}
To answer these questions, we evaluate various multi-task and meta-learning algorithms on the Meta-World benchmark. We include the training curves of all evaluations in Figure~\ref{fig:learningcurves} in the Appendix~\ref{app:curves}. Videos of the tasks and evaluations, along with all source code, are on the project webpage\footnote{Videos are on the project webpage,  at \url{meta-world.github.io}
%%SL: the period is omitted at the end of the URL intentionally (so that when someone copy & pastes it, they won't have to wonder why it's not working)
}.

In the multi-task evaluation, we evaluate the following RL algorithms:
\textbf{multi-task proximal policy optimization (PPO)} \cite{schulman2017proximal}: a policy gradient algorithm adapted to the multi-task setting by providing the one-hot task ID as input, \textbf{multi-task trust region policy optimization (TRPO)} \cite{schulman2015trust}: an on-policy policy gradient algorithm adapted to the multi-task setting using the one-hot task ID as input, \textbf{multi-task soft actor-critic (SAC)}~\cite{haarnoja2018soft}: an off-policy actor-critic algorithm adapted to the multi-task setting using the one-hot task ID as input, and an on-policy version of \textbf{task embeddings (TE)}~\cite{hausman2018learning}: a multi-task reinforcement learning algorithm that parameterizes the learned policies via shared skill embedding space.
For the meta-RL evaluation, we study three algorithms:
\textbf{RL$^2$} \cite{DBLP:journals/corr/DuanSCBSA16,wang1611learning}: an on-policy meta-RL algorithm that corresponds to training a GRU network with hidden states maintained across episodes within a task and trained with PPO, \textbf{model-agnostic meta-learning (MAML)} \cite{finn2017model, rothfuss2018promp}: an on-policy gradient-based meta-RL algorithm that embeds policy gradient steps into the meta-optimization, and is trained with PPO, and \textbf{probabilistic embeddings for actor-critic RL (PEARL)} \cite{rakelly2019efficient}: an off-policy actor-critic meta-RL algorithm, which learns to encode experience into a probabilistic embedding of the task that is fed to the actor and the critic. We use the baselines in the Garage \cite{garage} reinforcement learning library, which we developed for benchmarking Meta-World.

% \begin{wrapfigure}{4}{0.35\linewidth}
%     \centering
%     \vspace{-0.1cm}
%     \includegraphics[width=0.35\columnwidth]{figures_v2/per_env_success_ml1.pdf}
%     \vspace{-0.7cm}
%     \caption{\footnotesize{Comparison on our simplest meta-RL evaluation, ML1.}
%     }
%     \vspace{-0.4cm}
%     \label{fig:ml1}
% \end{wrapfigure}
We show results of the simplest meta-learning evaluation mode, ML1, in Figure~\ref{fig:ml1}. We find that there is room for improvement even in this very simple setting. Next, we look at results of multi-task learning across distinct tasks, starting with MT10 in Figure~\ref{fig:mt10} and in Table~\ref{tab:final_results}.
\\
We find that multi-task SAC is able to the learn the MT10 task suite well, achieving around 68\% success rate averaged across tasks, while multi-task PPO and TRPO are only able to achieve around a 30\% success rate. However, as we scale to 50 distict tasks with MT50, we find that MT-SAC and MT-PPO only achieve around a 35-38\% success rate, indicating that there is significant room for improvement in these methods
%%SL.10.11: I'm bothered by this evaluation because we have multi-head SAC, but presumably single-head TRPO and PPO, which is quite unfair.
%We run only multi-task soft actor-critic on MT50 due to computational constraints, but will include multi-task PPO and task embeddings in the final version of the paper.
%\todo{it would be really nice to say something about efficiency of these methods compared to individual single-task training.}

Finally, we study the ML10 and ML45 meta-learning benchmarks, which require learning the meta-training tasks and generalizing to new meta-test tasks with small amounts of experience. From Figure~\ref{fig:ml45} and Table~\ref{tab:final_results}, we find that the prior meta-RL methods, MAML and RL$^2$ reach 35\% and 31\% success on ML10 test tasks, while PEARL achieves only 13\% on ML10.
%%SL.10.11: When writing experimental results sections like this, it's a good idea to either explain the results or just say what they are without theatrics, this "unable to generalize" business doesn't strike me as good form. Could rephrase like this: On ML10, MAML can generalize to some degree, achieving an average success rate of 40\%. This may be due to the consistent gradient-based adaptation of the MAML method~\cite{}. In contrast, RL$^2$ and PEARL generalize poorly to the meta-test set on ML10.
On ML45, MAML and RL$^2$ solve around 39.9\% and 33.3\% of the meta-test tasks. Note that, on both ML10 and ML45, the meta-training performance of all methods also has considerable room for improvement, suggesting that optimization challenges are generally more severe in the meta-learning setting. The fact that some methods nonetheless exhibit meaningful generalization suggests that the ML10 and ML45 benchmarks are solvable, but challenging for current methods, leaving considerable room for improvement in future work.
\begin{figure}[H]
    \centering
    \includegraphics[width=\columnwidth]{figures_v2/per_env_success_ml1.pdf}
    \vspace{-0.5cm}
    \caption{\footnotesize{Comparison on our simplest meta-RL evaluation, ML1 on 10 seeds. RL$^2$ shows the strongest performance in generalization. Pearl shows the weakest performance, though this could be attributed to difficulty in training its task encoder}
    }
    \label{fig:ml1}
    % \vspace{-3cm}
\end{figure}
\begin{figure}[H]
    
    \includegraphics[width=\columnwidth]{figures_v2/per_task_success_mt10.pdf}
    \vspace{-1cm}
    \caption{Performance of the tested MTRL algorithms on 10 seeds. MT-SAC performs the best on MT-10, exhibiting the greatest sample efficiency and performance. For detailed plots of these algorithm's learning curves, see appendix \ref{app:curves}.}
    %Multi-task SAC performs the best on MT10 and MT50. MAML is the best on ML10. However, on ML45, PEARL is able to achieve more than 65\% success rate on test tasks in stark contrast to 0\% success rate on the test tasks in ML10, suggesting that larger number of meta-training tasks is beneficial for meta-RL algorithms.}
    \label{fig:mt10}
\end{figure}
\vspace{3cm}
\clearpage
\begin{figure}[H]  % {figure}[htb!]
    \includegraphics[width=\columnwidth]{figures_v2/per_task_success_ml10.pdf}
    \caption{Performance of the tested meta-RL algorithms on 10 seeds. RL$^2$ shows the highest performance on the training tasks (86.9\%), however its ability to generalize is not that much greater than MAML (35.8\% for RL$^2$ and 31.6\% for MAML).}
    \label{fig:ml10}
\end{figure}
\begin{figure}[H]
    \vspace{-1cm}
    \includegraphics[width=\columnwidth]{figures_v2/per_task_success_mt50.pdf}
    \vspace{-0.75cm}
    \caption{Performance of the tested MTRL algorithms on 10 seeds. In MT-10, MT-SAC showed the highest performance, however its performance does not scale to MT-50, the more difficult benchmark. MT-PPO exhibits the better performance in this benchmark.}
    \label{fig:mt50}
\end{figure}

\begin{figure}[H]
    \includegraphics[width=\columnwidth]{figures_v2/per_task_success_ml45.pdf}
    \caption{\footnotesize Average of maximum success rate for ML-45. Note that, even on the challenging ML-45 benchmark, current methods already exhibit some degree of generalization, but meta-training performance leaves considerable room for improvement, suggesting that future work could attain better performance on these benchmarks. Though PEARL has week training performance, it has comparable performance on test tasks. RL$^2$ has the highest  We also show the max average success rates for all benchmarks in Table \ref{tab:final_results}.}
    \label{fig:ml45}
\end{figure}

\begin{table*}[h]
  \centering
  \scriptsize
  \def\arraystretch{0.9}
%   \def\arraystretch{0.85}
%   \setlength{\tabcolsep}{0.3em}
  \setlength{\tabcolsep}{1em}
  \begin{tabularx}{0.43\linewidth}{ccll*{9}{c}}
  \toprule
 \multicolumn{1}{c}{Methods} & \multicolumn{1}{c}{MT10} & \multicolumn{1}{c}{MT50}\\
  % &\multicolumn{1}{c}{}  &\multicolumn{1}{c}{}\\
\midrule
  Multi-task PPO&   30.5\% & \textbf{35.4}\%\\
  Multi-task TRPO&   31.3\% & 21.0\%\\
  Task embeddings&   20.9\% & 11.8\%\\
 Multi-task SAC&   \textbf{68.3}\% & \textbf{38.5}\%\\
    \bottomrule
    \end{tabularx}
    \begin{tabularx}{0.5\linewidth}{ccll*{9}{c}}
  \toprule
 \multicolumn{1}{c}{\multirow{2}[4]{*}{Methods}} & \multicolumn{2}{c}{ML10} & \multicolumn{2}{c}{ML45}\\
    \cmidrule(lr){2-3} \cmidrule(lr){4-5} %\cmidrule(lr){6-10}
  % &\multicolumn{2}{c}{} & \multicolumn{2}{c}{}\\
     & \multicolumn{1}{c}{meta-train}  & \multicolumn{1}{c}{meta-test}  & \multicolumn{1}{c}{meta-train} & \multicolumn{1}{c}{meta-test}\\
  \midrule
  MAML&   44.4\%  &31.6\% & 40.7\%  &\textbf{39.9\%}\\
  RL$^2$&   \textbf{86.9\%}  & \textbf{35.8}\% & \textbf{70\%}  &33.3\%\\
  PEARL&   23.2\%  & 13\% & 14.5\%  & 22\%\\
    \bottomrule
    \end{tabularx}
     \caption{\footnotesize The average maximum success rate over all tasks for MT10, MT50, ML10, and ML45 on 10 seeds. The best performance in each benchmark is bolden. For MT10 and MT50, we show the average training success rate of multi-task SAC and multi-task PPO respectively outperform other methods. For ML10 and ML45, we show the meta-train and meta-test success rates. RL$^2$ achieves best meta-train performance in ML10 and ML45, while MAML and RL$2$ get the best generalization performance in ML10 and ML45 meta-test tasks respectively.
     }
    \label{tab:final_results}
\end{table*}

% \begin{table*}[ht!]
%   \centering
%   \scriptsize
%   \def\arraystretch{0.9}
% %   \def\arraystretch{0.85}
% %   \setlength{\tabcolsep}{0.3em}
%   \setlength{\tabcolsep}{1.5em}
%   \begin{tabularx}{0.97\linewidth}{ccll*{9}{c}}
%   \toprule
%  \multicolumn{1}{c}{\multirow{2}[4]{*}{Methods}} & \multicolumn{2}{c}{ML10} & \multicolumn{2}{c}{ML45}\\
%     \cmidrule(lr){2-3} \cmidrule(lr){4-5} \cmidrule(lr){6-10}
%   &\multicolumn{2}{c}{} & \multicolumn{2}{c}{}\\
%      & \multicolumn{1}{c}{meta-train}  & \multicolumn{1}{c}{meta-test}  & \multicolumn{1}{c}{meta-train} & \multicolumn{1}{c}{meta-test}\\
%   \midrule
%   MAML&   25\%  & \mybox{\textbf{36\%}} & 21.14\%  &23.93\%\\
%   RL$^2$&   \mybox{\textbf{50\%}}  & 10\% & \mybox{\textbf{43.18\%}}  &20\%\\
%   PEARL&   42.78\%  & 0\% & 11.36\%  &\mybox{\textbf{30\%}}\\
%     \bottomrule
%     \end{tabularx}
%      \caption{\footnotesize Average success rates over all tasks for MT10, MT50, ML10, and ML50. The best performance in each benchmark is boxed and bolden. For MT10 and MT50, we show the average training success rate and Multi-task Multi-headed SAC outperforms other methods. For ML10 and ML45, we show the meta-train and meta-test success rates. RL$^2$ achieves best meta-train performance in ML10 and ML45, while MAML and PEARL get the best generalization performance in ML10 and ML45 meta-test tasks respectively.
%      }
%     \label{tab:final_results}
% \end{table*}


\iffalse
\subsection{Analysis of Current Methods}

* cite ray interference, potentially a reason that meta-RL methods struggle is that the gradients between tasks are interfering, multitask SAC with multiple heads partially address this issue.

Gradient interference in current meta-RL methods -- study why they fail.
\fi

\section{Conclusion}


In this work, we identified a set of conditions that underlies major challenges in multi-task optimization: conflicting gradients, high positive curvature, and large gradient differences. We proposed a simple algorithm (PCGrad) to mitigate these challenges via ``gradient surgery.'' PCGrad provides a simple solution to mitigating gradient interference, which substantially improves optimization performance. We provide simple didactic examples and subsequently show significant improvement in optimization for a variety of multi-task supervised learning and reinforcement learning problems. We show that, when some optimization challenges of multi-task learning are alleviated by PCGrad, we can obtain hypothesized benefits in efficiency and asymptotic performance of multi-task settings.

While we studied multi-task supervised learning and multi-task reinforcement learning in this work, we suspect the problem of conflicting gradients to be prevalent in a range of other settings and applications, such as meta-learning, continual learning, multi-goal imitation learning~\citep{codevilla2018end}, and multi-task problems in natural language processing applications~\citep{mccann2018natural}. Due to its simplicity and model-agnostic nature, we expect that applying PCGrad in these domains to be a promising avenue for future investigation. Further, the general idea of gradient surgery may be an important ingredient for alleviating a broader class of optimization challenges in deep learning, such as the challenges in the stability challenges in two-player games~\citep{roth2017stabilizing} and multi-agent optimizations~\citep{nedic2009distributed}. We believe this work to be a step towards simple yet general techniques for addressing some of these challenges.

\section*{Broader Impact}
\paragraph{Applications and Benefits.}
Despite recent success, current deep learning and deep RL methods mostly focus on tackling a single specific task from scratch. Prior methods have proposed methods that can perform multiple tasks, but they often yield comparable or even higher data complexity compared to learning each task individually. Our method enables deep learning systems that mitigate inferences between differing tasks and thus achieves data-efficient multi-task learning. Since our method is general and simple to apply to various problems, there are many possible real-world applications, including but not limited to computer vision systems, autonomous driving, and robotics. For computer vision systems, our method can be used to develop algorithms that enable efficient classification, instance and semantics segmentation and object detection at the same time, which could improve performances of computer vision systems by reusing features obtained from each task and lead to a leap in real-world domains such as autonomous driving. For robotics, there are many situations where multi-task learning is needed. For example, surgical robots are required to perform a wide range of tasks such as stitching and removing tumour from the patient’s body. Kitchen robots should be able to complete multiple chores such as cooking and washing dishes at the same time. Hence, our work represents a step towards making multi-task reinforcement learning more applicable to those settings.

\paragraph{Risks.}
However, there are potential risks that apply to all machine learning and reinforcement learning systems including ours, including but not limited to safety, reward specification in RL which is often difficult to acquire in the real world, bias in supervised learning systems due to the composition of training data, and compute/data-intensive training procedures. For example, safety issues arise when autonomous driving cars fail to generalize to out-of-distribution data, which leads to crashing or even hurting people. Moreover, reward specification in RL is generally inaccessible in the real world, making RL unable to scale to real robots. In supervised learning domains, learned models could inherit the bias that exists in the training dataset. Furthermore, training procedures of ML models are generally compute/data-intensive, which cause inequitable access to these models. Our method is not immune to these risks. Hence, we encourage future research to design more robust and safe multi-task RL algorithms that can prevent unsafe behaviors. It is also important to push research in self-supervised and unsupervised multi-task RL in order to resolve the issue of reward specification. For supervised learning, we recommend researchers to publish their trained multi-task learning models to make access to those models equitable to everyone in field and develop new datasets that can mitigate biases and also be readily used in multi-task learning.


%
%
%

\begin{ack}

The authors would like to thank Annie Xie for reviewing an earlier draft of the paper, Eric Mitchell for technical guidance, and Aravind Rajeswaran and Deirdre Quillen for helpful discussions. Tianhe Yu is partially supported by Intel Corporation. Saurabh Kumar is supported by an NSF Graduate Research Fellowship and the Stanford Knight Hennessy Fellowship. Abhishek Gupta is supported by an NSF Graduate Research Fellowship. Chelsea Finn is a CIFAR Fellow in the Learning in Machines and Brains program.

\end{ack}

\bibliography{reference}
\bibliographystyle{plainnat}

\newpage
{\Large \bf Appendix}
\appendix
\def\rvgone{{\mathbf{g_1}}}
\def\rvgtwo{{\mathbf{g_2}}}



\section{Proofs}
\subsection{Proof of Theorem 1}
\label{app:proof}
\theoremconvergence*

\begin{proof}

We will use the shorthand $|| \cdot ||$ to denote the $L_2$-norm and $\nabla \loss = \nabla_\theta \loss$, where $\theta$ is the parameter vector. Following Definition~\ref{def:angle} and \ref{def:setup}, let $\rvgone = \nabla \loss_1$, $\rvgtwo = \nabla \loss_2$, $\rvg = \nabla\loss = \rvgone + \rvgtwo$, and $\phi_{12}$ be the angle between $\rvgone$ and $\rvgtwo$.

At each PCGrad update, we have two cases: $cos(\phi_{12}) \geq 0$ or $\cos(\phi_{12} ) < 0$.

If $\cos(\phi_{12}) \geq 0$, then we apply the standard gradient descent update using $t \leq \frac{1}{L}$, which leads to a strict decrease in the objective function value $\loss(\theta)$ (since it is also convex) unless $\nabla \loss(\theta) = 0$, which occurs only when $\theta = \theta^*$~\citep{boyd2004convex}. 

In the case that $\cos(\phi_{12}) < 0$, we proceed as follows:

Our assumption that $\nabla \loss$ is Lipschitz continuous with constant $L$ implies that $\nabla^2 \loss(\theta) - LI$ is a negative semi-definite matrix. Using this fact, we can perform a quadratic expansion of $\loss$ around $\loss(\theta)$ and obtain the following inequality:
\begin{align*}
\loss(\theta^+) &\leq \loss(\theta) + \nabla \loss(\theta)^T (\theta^+ - \theta) + \frac{1}{2} \nabla^2 \loss(\theta) ||\theta^+ - \theta||^2 \\
&\leq \loss(\theta) + \nabla \loss(\theta)^T (\theta^+ - \theta) + \frac{1}{2} L ||\theta^+ - \theta||^2
\end{align*}
Now, we can plug in the PCGrad update by letting $\theta^+ = \theta - t\cdot(\rvg - \frac{\rvgone \cdot \rvgtwo}{||\rvgone||^2}\rvgone - \frac{\rvgone \cdot \rvgtwo}{||\rvgtwo||^2}\rvgtwo)$. We then get:
\begin{align}
    \loss(\theta^+) &\leq \loss(\theta) + t\cdot\rvg^T (-\rvg + \frac{\rvgone \cdot \rvgtwo}{||\rvgone||^2}\rvgone + \frac{\rvgone \cdot \rvgtwo}{||\rvgtwo||^2}\rvgtwo) 
    + \frac{1}{2}Lt^2||\rvg - \frac{\rvgone \cdot \rvgtwo}{||\rvgone||^2}\rvgone - \frac{\rvgone \cdot \rvgtwo}{||\rvgtwo||^2}\rvgtwo||^2\nonumber \\ \nonumber\\
    &\text{(Expanding, using the identity $\rvg = \rvgone + \rvgtwo$)}\nonumber
    \\ \nonumber\\
    &= \loss(\theta) + t\left(-||\rvgone||^2 - ||\rvgtwo||^2 + \frac{(\rvgone \cdot \rvgtwo)^2}{||\rvgone||^2} + \frac{(\rvgone \cdot \rvgtwo)^2}{||\rvgtwo||^2}\right)
    + \frac{1}{2}Lt^2||\rvgone + \rvgtwo \nonumber\\ 
    &- \frac{\rvgone \cdot \rvgtwo}{||\rvgone||^2}\rvgone - \frac{\rvgone \cdot \rvgtwo}{||\rvgtwo||^2}\rvgtwo||^2\nonumber
    \\ \nonumber\\
    &\text{(Expanding further and re-arranging terms)}\nonumber
    \\ \nonumber\\
    &= \loss(\theta) - (t - \frac{1}{2}Lt^2)(||\rvgone||^2 + ||\rvgtwo||^2 - \frac{(\rvgone \cdot \rvgtwo)^2}{||\rvgone||^2} - \frac{(\rvgone \cdot \rvgtwo)^2}{||\rvgtwo||^2})\nonumber\\
    &- Lt^2(\rvgone \cdot \rvgtwo - \frac{(\rvgone \cdot \rvgtwo)^2}{||\rvgone||^2 ||\rvgtwo||^2}\rvgone \cdot \rvgtwo) \nonumber
    \\ \nonumber\\
    &\text{(Using the identity $\cos(\phi_{12}) = \frac{\rvgone \cdot \rvgtwo}{||\rvgone|| ||\rvgtwo||}$)}\nonumber
    \\ \nonumber\\
    &= \loss(\theta) - (t  - \frac{1}{2}Lt^2) [(1 - \cos^2(\phi_{12})) ||\rvgone||^2 + (1 - \cos^2(\phi_{12})) ||\rvgtwo||^2 ]\nonumber \\
    &- Lt^2 (1 - \cos^2(\phi_{12})) ||\rvgone|| ||\rvgtwo|| \cos(\phi_{12}) \label{eq:pcgrad_bound}\\
    &\text{(Note that $\cos(\phi_{12}) < 0$ so the final term is non-negative)}\nonumber
\end{align}
Using $t \leq \frac{1}{L}$, we know that $-(1 - \frac{1}{2} Lt) = \frac{1}{2}Lt - 1 \leq \frac{1}{2}L(1/L) - 1 = \frac{-1}{2}$ and $Lt^2 \leq t$. 

Plugging this into the last expression above, we can conclude the following:
\begin{align*}
    \loss(\theta^+) &\leq \loss(\theta) - \frac{1}{2}t [(1 - \cos^2(\phi_{12})) ||\rvgone||^2 + (1 - \cos^2(\phi_{12})) ||\rvgtwo||^2 ]\\
    &- t (1 - \cos^2(\phi_{12})) ||\rvgone|| ||\rvgtwo|| \cos(\phi_{12}) \\
    &= \loss(\theta) - \frac{1}{2}t (1 - \cos^2(\phi_{12})) [ ||\rvgone||^2 + 
    2 ||\rvgone|| ||\rvgtwo|| \cos(\phi_{12}) +
    ||\rvgtwo||^2 ] \\
    &= \loss(\theta) - \frac{1}{2}t (1 - \cos^2(\phi_{12})) [ ||\rvgone||^2 + 
    2 \rvgone \cdot \rvgtwo +
    ||\rvgtwo||^2 ] \\
    &= \loss(\theta) - \frac{1}{2}t (1 - \cos^2(\phi_{12})) ||\rvgone + \rvgtwo||^2 \\
    &= \loss(\theta) - \frac{1}{2}t (1 - \cos^2(\phi_{12})) \|\rvg\|^2
\end{align*}

If $\cos(\phi_{12}) > -1$, then $\frac{1}{2}t (1 - \cos^2(\phi_{12})) \|\rvg\|^2$ will always be positive unless $\rvg = 0$. This inequality implies that the objective function
value strictly decreases with each iteration where $\cos(\phi_{12}) > -1$.

Hence repeatedly applying PCGrad process can either reach the optimal value $\loss(\theta) = \loss(\theta^*)$ or $\cos(\phi_{12}) = -1$, in which case $\frac{1}{2}t (1 - \cos^2(\phi_{12})) \|\rvg\|^2 = 0$. Note that this result only holds when we choose $t$ to be small enough, i.e. $t \leq \frac{1}{L}$.

\end{proof}

\begin{corollary}
\label{cor:moretasks}
Assume the $n$ objectives $\loss_1, \loss_2, ..., \loss_n$ are convex and differentiable. Suppose the gradient of $\loss$ is Lipschitz continuous with constant $L > 0$. Assume that $\cos(\rvg, \rvg^\text{PC}) \geq \frac{1}{2}$. 
Then, the PCGrad update rule with step size $t \leq \frac{1}{L}$ will converge to either (1) a location in the optimization landscape where $\cos(\rvg_i, \rvg_j) = -1 \forall i, j$ or (2) the optimal value $\loss(\theta^*)$. 
\end{corollary}
\begin{proof}
Our assumption that $\nabla \loss$ is Lipschitz continuous with constant $L$ implies that $\nabla^2 \loss(\theta) - LI$ is a negative semi-definite matrix. Using this fact, we can perform a quadratic expansion of $\loss$ around $\loss(\theta)$ and obtain the following inequality:
\begin{align*}
\loss(\theta^+) &\leq \loss(\theta) + \nabla \loss(\theta)^T (\theta^+ - \theta) + \frac{1}{2} \nabla^2 \loss(\theta) ||\theta^+ - \theta||^2 \\
&\leq \loss(\theta) + \nabla \loss(\theta)^T (\theta^+ - \theta) + \frac{1}{2} L ||\theta^+ - \theta||^2
\end{align*}
Now, we can plug in the PCGrad update by letting $\theta^+ = \theta - t\cdot \rvg^\text{PC}$. We then get:
\begin{align*}
    \loss(\theta^+) &\leq \loss(\theta) - t\cdot\rvg^T \rvg^\text{PC}
    + \frac{1}{2}Lt^2||\rvg^\text{PC}||^2\nonumber \\
    &\text{(Using the assumption that $\cos(\rvg, \rvg^\text{PC}) \geq \frac{1}{2}$.)} \\
    &\leq \loss(\theta) - \frac{1}{2} t||\rvg||\cdot||\rvg^\text{PC}||
    + \frac{1}{2}Lt^2||\rvg^\text{PC}||^2\nonumber \\
    &\leq \loss(\theta) - \frac{1}{2} t||\rvg||\cdot||\rvg^\text{PC}||
    + \frac{1}{2}Lt^2||\rvg^\text{PC}|| \cdot ||\rvg|| \nonumber \nonumber \\
\end{align*}
Note that $-\frac{1}{2} t ||\rvg||\cdot||\rvg^\text{PC}||
    + \frac{1}{2}Lt^2||\rvg^\text{PC}|| \cdot ||\rvg|| \leq 0$ when $t \leq \frac{1}{L}$. Further, when $t < \frac{1}{L}$, $-\frac{1}{2} t ||\rvg||\cdot||\rvg^\text{PC}||
    + \frac{1}{2}Lt^2||\rvg^\text{PC}|| \cdot ||\rvg|| = 0$ if and only if $||\rvg|| = 0$ or $||\rvg^\text{PC}||=0$. 
    
Hence repeatedly applying PCGrad process can either reach the optimal value $\loss(\theta) = \loss(\theta^*)$ or a location in the optimization landscape where $\cos(\rvg_i, \rvg_j) = -1$ for all pairs of tasks $i, j$. Note that this result only holds when we choose $t$ to be small enough, i.e. $t \leq \frac{1}{L}$.
\end{proof}

\begin{proposition}
\label{prop:nonconvex}
Assume $\loss_1$ and $\loss_2$ are differentiable but possibly non-convex. Suppose the gradient of $\loss$ is Lipschitz continuous with constant $L > 0$.
Then, the PCGrad update rule with step size $t \leq \frac{1}{L}$ will converge to either (1) a location in the optimization landscape where $\cos(\phi_{12}) = -1$ or (2) find a $\theta_k$ that is almost a stationary point.
\end{proposition}
\begin{proof}
Following Definition~\ref{def:angle} and \ref{def:setup}, let $\rvgone_k = \nabla \loss_1$ at iteration $k$, $\rvgtwo_k = \nabla \loss_2$ at iteration $k$, and $\rvg_k = \nabla\loss = \rvgone_k + \rvgtwo_k$ at iteration $k$, and $\phi_{12, k}$ be the angle between $\rvgone_k$ and $\rvgtwo_k$.


From the proof of Theorem 1, when $\cos(\phi_{12},k) < 0$ we have:
$$
||\rvg_k||^2 \leq \frac{2}{t} \frac{\loss(\theta_{k-1}) - \loss(\theta_{k})}{(1 - \cos^2(\phi_{12, k}))}.
$$

Thus, we have:
\begin{align*}
\min_{0 \leq k \leq K} || \rvg_k ||^2 &\leq \frac{1}{K} \sum_{i = 0}^{K-1} ||\rvg_i||^2 \\
&\leq \frac{2}{Kt} \sum_{i = 0}^{K-1} \frac{\loss(\theta_{i-1}) - \loss(\theta_{i})}{(1 - \cos^2(\phi_{12, i}))}
\end{align*}

If at any iteration, $\cos(\phi_{12, k}) = -1$, then the optimization will stop at that point. If $\forall k \in [0, K]$, $\cos(\phi_{12, k}) \geq \alpha > -1$, then, we have:

\begin{align*}
\min_{0 \leq k \leq K} || \rvg_k ||^2 &\leq \frac{2}{K (1 - \alpha^2) t} \sum_{i = 0}^{K-1} (\loss(\theta_{i-1}) - \loss(\theta_{i})) \\
&= \frac{2}{K (1 - \alpha^2) t} (\loss(\theta_{0}) - \loss(\theta_{K})) \\
&\leq \frac{2}{K (1 - \alpha^2) t} (\loss(\theta_{0}) - \loss^*).
\end{align*}
where $\loss^*$ is the minimal function value.
\end{proof}

Note that the convergence rate of PCGrad in the non-convex setting largely depends on the value of $\alpha$ and generally how small $\cos(\phi_{12, k})$ is on average.

\subsection{Proof of Theorem 2}
\label{app:proof2}

\theoremlocal*

\begin{proof}

Note that $\theta^{\text{MT}} = \theta - t\cdot\rvg$ and $\theta^{\text{PCGrad}} = \theta - t(\rvg - \frac{\rvgone \cdot \rvgtwo}{||\rvgone||^2}\rvgone - \frac{\rvgone \cdot \rvgtwo}{||\rvgtwo||^2}\rvgtwo)$. Based on the condition that $\mathbf{H}(\loss; \theta, \theta^{\text{MT}}) \geq \ell\|\rvg\|_2^2$, we first apply Taylor's Theorem to $\loss(\theta^{\text{MT}})$ and obtain the following result:
\begin{align}
    \loss(\theta^{\text{MT}}) &= \loss(\theta) + \rvg^T(-t\rvg) + \int_0^1 (-t\rvg)^T\frac{\nabla^2\loss(\theta+a\cdot(-t\rvg))}{2}(-t\rvg)da \nonumber\\
    &\geq \loss(\theta) + \rvg^T(-t\rvg) + t^2\cdot \frac{1}{2}\ell\cdot\|\rvg\|_2^2 \nonumber\\
    &= \loss(\theta) - t\|\rvg\|_2^2 + \frac{1}{2}\ell t^2\|\rvg\|_2^2 \nonumber\\
    &= \loss(\theta) + (\frac{1}{2}\ell t^2 - t)\|\rvg\|_2^2 \label{eq:mt_bound}
\end{align}
where the first inequality follows from Definition~\ref{def:curvature} and the assumption $\mathbf{H}(\loss; \theta, \theta^{\text{MT}}) \geq \ell\|\rvg\|_2^2$. From equation~\ref{eq:pcgrad_bound}, we have the simplified upper bound for $\loss(\theta^{\text{PCGrad}})$:
\begin{align}
    \loss(\theta^{\text{PCGrad}}) &\leq \loss(\theta) - (1-\cos^2\phi_{12})[(t-\frac{1}{2}Lt^2)\cdot(\|\rvgone\|_2^2+\|\rvgone\|_2^2) + Lt^2\|\rvgone\|_2\|\rvgtwo\|_2\cos\phi_{12}]\label{eq:pcgrad_bound_simplified}
\end{align}
Apply Equation~\ref{eq:mt_bound} and Equation~\ref{eq:pcgrad_bound_simplified} and we have the following inequality:
\begin{align}
    &\loss(\theta^{\text{MT}}) - \loss(\theta^{\text{PCGrad}}) \geq \loss(\theta) + (\frac{1}{2}\ell t^2 - t)\|\rvg\|_2^2 - \loss(\theta)\nonumber\\
    & + (1-\cos^2\phi_{12})[(t-\frac{1}{2}Lt^2)(\|\rvgone\|_2^2+\|\rvgtwo\|_2^2) + Lt^2\|\rvgone\|_2\|\rvgtwo\|_2\cos\phi_{12}] \nonumber\\
    &= (\frac{1}{2}\ell t^2 - t)\|\rvgone + \rvgtwo\|_2^2\!\!+\!\! (1-\cos^2\phi_{12})\!\!\left[(t-\frac{1}{2}Lt^2)\!\cdot\!(\|\rvgone\|_2^2\!+\!\|\rvgtwo\|_2^2)\!+\! Lt^2\|\rvgone\|_2\|\rvgtwo\|_2\cos\phi_{12}\right]\nonumber\\
    &= \left(\frac{1}{2}\|\rvgone+\rvgtwo\|_2^2\ell - \frac{1-\cos^2\phi_{12}}{2}(\|\rvgone\|_2^2+\|\rvgtwo\|_2^2 - 2\|\rvgone\|_2\|\rvgtwo\|_2\cos\phi_{12})L\right)t^2\nonumber\\
    &- \left((\|\rvgone\|_2^2+\|\rvgtwo\|_2^2)\cos^2\phi_{12} + 2\|\rvgone\|_2\|\rvgtwo\|_2\cos\phi_{12}\right)t\nonumber\\
    &= \left(\frac{1}{2}\|\rvgone+\rvgtwo\|_2^2\ell - \frac{1-\cos^2\phi_{12}}{2}\|\rvgone-\rvgtwo\|_2^2L\right)t^2\nonumber\\
    &- \left((\|\rvgone\|_2^2+\|\rvgtwo\|_2^2)\cos^2\phi_{12} + 2\|\rvgone\|_2\|\rvgtwo\|_2\cos\phi_{12}\right)t\nonumber\\
    &= t\cdot\left[\left(\frac{1}{2}\|\rvgone+\rvgtwo\|_2^2\ell - \frac{1-\cos^2\phi_{12}}{2}(\|\rvgone-\rvgtwo\|_2^2)L\right)t\right.\nonumber\\
    &\left.- \left((\|\rvgone\|_2^2+\|\rvgtwo\|_2^2)\cos^2\phi_{12} + 2\|\rvgone\|_2\|\rvgtwo\|_2\cos\phi_{12}\right)\right]\label{eq:difference}
\end{align}
Since $\cos\phi_{12} \leq -\Phi(\rvgone, \rvgtwo) =  -\frac{2\|\rvgone\|_2\|\rvgtwo\|_2}{\|\rvgone\|_2^2 + \|\rvgtwo\|_2^2}$ and $\ell \geq \xi(\rvgone, \rvgtwo) = \frac{(1 - \cos^2\phi_{12})(\|\rvgone-\rvgtwo\|_2^2)}{\|\rvgone + \rvgtwo\|_2^2}L$, we have $$\frac{1}{2}\|\rvgone+\rvgtwo\|_2^2\ell - \frac{1-\cos^2\phi_{12}}{2}\|\rvgone-\rvgtwo\|_2^2L \geq 0$$ and $$(\|\rvgone\|_2^2+\|\rvgtwo\|_2^2)\cos^2\phi_{12} + 2\|\rvgone\|_2\|\rvgtwo\|_2\cos\phi_{12} \geq 0.$$ By the condition that $t \geq \frac{2}{\ell - \xi(\rvgone, \rvgtwo)L} = \frac{2}{\ell - \frac{(1 - \cos^2\phi_{12})\|\rvgone-\rvgtwo\|_2^2}{\|\rvgone + \rvgtwo\|_2^2}L}$ and monotonicity of linear functions, we have the following:
\begin{align}
    &\loss(\theta^{\text{MT}})\!-\!\loss(\theta^{\text{PCGrad}}) \geq [\left(\frac{1}{2}\|\rvgone\!+\!\rvgtwo\|_2^2\ell-\frac{1\!-\!\cos^2\phi_{12}}{2}\!\cdot\!\|\rvgone\!-\!\rvgtwo\|_2^2L\right)\cdot\frac{2}{\ell\!-\! \frac{(1-\cos^2\phi_{12})\|\rvgone-\rvgtwo\|_2^2}{\|\rvgone + \rvgtwo\|_2^2}L}\nonumber\\
    &- \left((\|\rvgone\|_2^2+\|\rvgtwo\|_2^2)\cos^2\phi_{12} + 2\|\rvgone\|_2\|\rvgtwo\|_2\cos\phi_{12}\right)]\cdot t\nonumber\\
    &=[\|\rvgone+\rvgtwo\|_2^2\cdot\left(\ell-\frac{(1-\cos^2\phi_{12})\cdot\|\rvgone-\rvgtwo\|_2^2}{\|\rvgone+\rvgtwo\|_2^2})L\right)\cdot\frac{1}{\ell - \frac{(1 - \cos^2\phi_{12})\|\rvgone-\rvgtwo\|_2^2}{\|\rvgone + \rvgtwo\|_2^2}L}\nonumber\\
    &- \left((\|\rvgone\|_2^2+\|\rvgtwo\|_2^2)\cos^2\phi_{12} + 2\|\rvgone\|_2\|\rvgtwo\|_2\cos\phi_{12}\right)]\cdot t\nonumber\\
    &= \left[\|\rvgone+\rvgtwo\|_2^2 - \left((\|\rvgone\|_2^2+\|\rvgtwo\|_2^2)\cos^2\phi_{12}+ 2\|\rvgone\|_2\|\rvgtwo\|_2\cos\phi_{12}\right)\right]\cdot t\nonumber\\
    &=\left[\|\rvgone\|_2^2\!+\!\|\rvgtwo\|_2^2\!+\!2\|\rvgone\|_2\|\rvgtwo\|_2\!\cos\phi_{12}\!-\! \left((\|\rvgone\|_2^2\!+\!\|\rvgtwo\|_2^2)\cos^2\!\phi_{12}\! +\! 2\|\rvgone\|_2\|\rvgtwo\|_2\!\cos\phi_{12}\right)\right]\cdot t\nonumber\\
    &= (1-\cos^2\phi_{12})(\|\rvgone\|_2^2+\|\rvgtwo\|_2^2)\cdot t\nonumber\\
    &\geq 0\nonumber
\end{align}
\end{proof}

\subsection{PCGrad: Sufficient and Necessary Conditions for Loss Improvement}
\label{app:theorem3}

Beyond the sufficient conditions shown in Theorem~\ref{thm:local}, we also present the sufficient and necessary conditions under which PCGrad achieves lower loss after one gradient update in Theorem~\ref{thm:necessary} in the two-task setting.

\begin{restatable}{theorem}{theoremnecessary}
\label{thm:necessary}
Suppose $\loss$ is differentiable and the gradient of $\loss$ is Lipschitz continuous with constant $L > 0$. 
Let $\theta^{\text{MT}}$ and $\theta^{\text{PCGrad}}$ be the parameters after applying one update to $\theta$ with $\rvg$ and PCGrad-modified gradient $\rvg^{\text{PC}}$ respectively, with step size $t > 0$. 
Moreover, assume $\mathbf{H}(\loss; \theta, \theta^{\text{MT}}) \geq \ell\|\rvg\|_2^2$ for some constant $\ell \leq L$, i.e. the multi-task curvature is lower-bounded. Then $\loss(\theta^{\text{PCGrad}}) \leq \loss(\theta^{\text{MT}})$ if and only if 
\begin{itemize}
    \setlength\itemsep{1em}
    \item $-\Phi(\rvgone, \rvgtwo) \leq \cos\phi_{12} < 0$
    \item $\ell \leq \xi(\rvgone, \rvgtwo)L$
    \item $0 < t \leq \frac{(\|\rvgone\|^2_2 + \|\rvgtwo\|^2_2)\cos^2\phi_{12} + 2\|\rvgone\|_2\|\rvgtwo\|_2\cos\phi_{12}}{\frac{1}{2}\|\rvgone + \rvgtwo\|_2^2\ell - \frac{1 - \cos^2\phi_{12}}{2}(\|\rvgone-\rvgtwo\|_2^2)L}$
\end{itemize} or
\begin{itemize}
    \setlength\itemsep{1em}
    \item $\cos\phi_{12} \leq -\Phi(\rvgone, \rvgtwo)$
    \item $\ell \geq \xi(\rvgone, \rvgtwo)L$
    \item $t \geq \frac{(\|\rvgone\|^2_2 + \|\rvgtwo\|^2_2)\cos^2\phi_{12} + 2\|\rvgone\|_2\|\rvgtwo\|_2\cos\phi_{12}}{\frac{1}{2}\|\rvgone + \rvgtwo\|_2^2\ell - \frac{1 - \cos^2\phi_{12}}{2}(\|\rvgone-\rvgtwo\|_2^2)L}.$
\end{itemize}
\end{restatable}

\vspace{-0.3cm}
\begin{proof}
To show the necessary conditions, from Equation~\ref{eq:difference}, all we need is
\begin{align}
    &t\cdot[(\frac{1}{2}\|\rvgone+\rvgtwo\|_2^2\ell - \frac{1-\cos^2\phi_{12}}{2}(\|\rvgone-\rvgtwo\|_2^2)L)t\nonumber\\
    &- \left((\|\rvgone\|_2^2+\|\rvgtwo\|_2^2)\cos^2\phi_{12} + 2\|\rvgone\|_2\|\rvgtwo\|_2\cos\phi_{12}\right)] \geq 0
\end{align}
Since $t \geq 0$, it reduces to show
\begin{align}
    &(\frac{1}{2}\|\rvgone+\rvgtwo\|_2^2\ell - \frac{1-\cos^2\phi_{12}}{2}(\|\rvgone-\rvgtwo\|_2^2)L)t\nonumber\\
    &- \left((\|\rvgone\|_2^2+\|\rvgtwo\|_2^2)\cos^2\phi_{12} + 2\|\rvgone\|_2\|\rvgtwo\|_2\cos\phi_{12}\right) \geq 0\label{eq:necessary_cond}
\end{align}
For Equation~\ref{eq:necessary_cond} to hold while ensuring that $t \geq 0$, there are two cases:
\begin{itemize}
    \item $\frac{1}{2}\|\rvgone+\rvgtwo\|_2^2\ell - (1-\cos^2\phi_{12})(\|\rvgone\|_2^2+\|\rvgtwo\|_2^2)L \geq 0$,\\
    $(\|\rvgone\|_2^2+\|\rvgtwo\|_2^2)\cos^2\phi_{12} + 2\|\rvgone\|_2\|\rvgtwo\|_2\cos\phi_{12} \geq 0$,\\
    $t \geq \frac{(\|\rvgone\|^2_2 + \|\rvgtwo\|^2_2)\cos^2\phi_{12} + 2\|\rvgone\|_2\|\rvgtwo\|_2\cos\phi_{12}}{\frac{1}{2}\|\rvgone + \rvgtwo\|_2^2\ell - \frac{1 - \cos^2\phi_{12}}{2}(\|\rvgone-\rvgtwo\|_2^2)L}$
    \item $\frac{1}{2}\|\rvgone+\rvgtwo\|_2^2\ell - (1-\cos^2\phi_{12})(\|\rvgone\|_2^2+\|\rvgtwo\|_2^2)L \leq 0$,\\
    $(\|\rvgone\|_2^2+\|\rvgtwo\|_2^2)\cos^2\phi_{12} + 2\|\rvgone\|_2\|\rvgtwo\|_2\cos\phi_{12} \leq 0$,\\
    $t \geq \frac{(\|\rvgone\|^2_2 + \|\rvgtwo\|^2_2)\cos^2\phi_{12} + 2\|\rvgone\|_2\|\rvgtwo\|_2\cos\phi_{12}}{\frac{1}{2}\|\rvgone + \rvgtwo\|_2^2\ell - \frac{1 - \cos^2\phi_{12}}{2}(\|\rvgone-\rvgtwo\|_2^2)L}$
\end{itemize}
, which can be simplified to
\begin{itemize}
    \item $\cos\phi_{12} \leq -\frac{2\|\rvgone\|_2\|\rvgtwo\|_2}{\|\rvgone\|_2^2 + \|\rvgtwo\|_2^2} = -\Phi(\rvgone, \rvgtwo)$,\\
    $\ell \geq \frac{(1 - \cos^2\phi_{12})(\|\rvgone\|_2^2 + \|\rvgtwo\|_2^2)}{\|\rvgone + \rvgtwo\|_2^2}L = \xi(\rvgone, \rvgtwo)$,\\
    $t \geq \frac{(\|\rvgone\|^2_2 + \|\rvgtwo\|^2_2)\cos^2\phi_{12} + 2\|\rvgone\|_2\|\rvgtwo\|_2\cos\phi_{12}}{\frac{1}{2}\|\rvgone + \rvgtwo\|_2^2\ell - \frac{1 - \cos^2\phi_{12}}{2}(\|\rvgone-\rvgtwo\|_2^2)L}$
    \item $-\frac{2\|\rvgone\|_2\|\rvgtwo\|_2}{\|\rvgone\|_2^2 + \|\rvgtwo\|_2^2} = -\Phi(\rvgone, \rvgtwo) \leq \cos\phi_{12} < 0$,\\
    $\ell \leq \frac{(1 - \cos^2\phi_{12})(\|\rvgone\|_2^2 + \|\rvgtwo\|_2^2)}{\|\rvgone + \rvgtwo\|_2^2}L = \xi(\rvgone, \rvgtwo)$,\\
    $0 < t \leq \frac{(\|\rvgone\|^2_2 + \|\rvgtwo\|^2_2)\cos^2\phi_{12} + 2\|\rvgone\|_2\|\rvgtwo\|_2\cos\phi_{12}}{\frac{1}{2}\|\rvgone + \rvgtwo\|_2^2\ell - \frac{1 - \cos^2\phi_{12}}{2}(\|\rvgone-\rvgtwo\|_2^2)L}$.
\end{itemize}

The sufficient conditions hold as we can plug the conditions to RHS of Equation~\ref{eq:necessary_cond} and achieve non-negative result.
\end{proof}
\vspace{-0.3cm}

\subsection{Convergence of PCGrad with Momentum-Based Gradient Descent}
\label{app:momentum}

In this subsection, we show convergence of PCGrad with momentum-based methods, which is more aligned with our practical implementation. In our analysis, we consider the heavy ball method~\cite{polyak1964some} as follows: $$\theta_{k+1} \leftarrow \theta_k - \alpha_k \nabla\loss(\theta_k) + \beta_k (\theta_k - \theta_{k-1})$$ where $k$ denotes the $k$-th step and $\alpha_k$ and $\beta_k$ are step sizes for the gradient and momentum at step $k$ respectively. We now present our theorem.

\begin{restatable}{theorem}{momentum}
\label{thm:momentum}
Assume $\loss_1$ and $\loss_2$ are $\mu_1$- and $\mu_2$-strongly convex and also $L_1$- and $L_2$-smooth respectively where $\mu_1, \mu_2, L_1, L_2 > 0$. Define $\phi^k_{12}$ as the angle between two task gradients $\rvgone{(\theta_k)}$ and $\rvgtwo{(\theta_k)}$ and define $R_k = \frac{\|\rvgone(\theta_k)\|}{\|\rvgtwo(\theta_k)\|}$. Denote $\mu_k = (1 - \cos\phi^k_{12}/R_k)\mu_1 + (1 - \cos\phi^k_{12}\cdot R_k)\mu_2$ and $L_k = (1 - \cos\phi^k_{12}/R_k)L_1 + (1 - \cos\phi^k_{12}\cdot R_k)L_2$
Then, the PCGrad update rule of the heavy ball method with step sizes $\alpha_k = \frac{4}{\sqrt{L_k} + \sqrt{\mu_k}}$ and $\beta_k = \max\{|1 - \sqrt{\alpha_k\mu_k}|,|1 - \sqrt{\alpha_kL_k}|\}^2$ will converge linearly to either (1) a location in the optimization landscape where $\cos(\phi^k_{12}) = -1$ or (2) the optimal value $\loss(\theta^*)$.
\end{restatable}

\begin{proof}
We first observe that the PCGrad-modified gradient $\rvg^{\text{PC}}$ has the following identity:
\begin{align}
    \rvg^\text{PC} &= \rvg - \frac{\rvgone \cdot \rvgtwo}{||\rvgone||^2}\rvgone - \frac{\rvgone \cdot \rvgtwo}{||\rvgtwo||^2}\rvgtwo\nonumber\\
    &= (1 - \frac{\rvgone \cdot \rvgtwo}{||\rvgone||^2})\rvgone + (1 - \frac{\rvgone \cdot \rvgtwo}{||\rvgtwo||^2})\rvgtwo\nonumber\\
    &= (1 - \frac{\rvgone \cdot \rvgtwo}{\|\rvgone\|\|\rvgtwo\|}\frac{\|\rvgtwo\|}{\|\rvgone\|})\rvgone + (1 - \frac{\rvgone \cdot \rvgtwo}{\|\rvgone\|\|\rvgtwo\|}\frac{\|\rvgone\|}{\|\rvgtwo\|})\rvgtwo\nonumber\\
    &= (1 - \cos\phi_{12} / R)\rvgone + (1 - \cos\phi_{12}\cdot R)\rvgtwo\label{eq:split}.
\end{align}
Applying Equation~\ref{eq:split}, we can write the PCGrad update rule of the heavy ball method in matrix form as follows:
\begin{align*}
    \left\|\begin{bmatrix}
    \theta_{k+1} - \theta^*\\
    \theta_k - \theta^*
    \end{bmatrix}\right\|_2 &= \left\|\begin{bmatrix}
    \theta_{k} + \beta_k(\theta_k - \theta_{k-1}) -  \theta^* \\
    \theta_k - \theta^*
    \end{bmatrix} -  \alpha_k\begin{bmatrix}
    \rvg^{\text{PC}}(\theta_k) \\
    0
    \end{bmatrix}\right\|_2\\
    &= \left\|\begin{bmatrix}
    \theta_{k} + \beta_k(\theta_k - \theta_{k-1}) -  \theta^* \\
    \theta_k - \theta^*
    \end{bmatrix}\right.\\
    &\left.-  \alpha_k\begin{bmatrix}
    (1 - \cos\phi^k_{12}/R_k)\rvgone(\theta_k) + (1 - \cos\phi^k_{12}\cdot R_k)\rvgtwo(\theta_k) \\
    0
    \end{bmatrix}\right\|_2\\
    &= \left\|\begin{bmatrix}
    \theta_{k} + \beta_k(\theta_k - \theta_{k-1}) -  \theta^* \\
    \theta_k - \theta^*
    \end{bmatrix}\right.\\
    &\left.- \alpha_k\begin{bmatrix}
    \left[(1 - \cos\phi^k_{12}/R_k)\nabla^2\mathcal{L}_1(z_k) + (1 - \cos\phi^k_{12}\cdot R_k)\nabla^2\mathcal{L}_2(z'_k)\right](\theta_k -  \theta^*) \\
    0
    \end{bmatrix}\right\|_2\\
    &\text{for some }z_k,  z'_k\text{ on the line segment between }\theta_k\text{ and }\theta^*\\
    &=\left\|\begin{bmatrix}
    (1+\beta_k)I-\alpha_kH_k & -\beta_k I\\
    I & 0
    \end{bmatrix}\begin{bmatrix}
    \theta_k -  \theta^*\\
    \theta_{k-1} - \theta^*
    \end{bmatrix}\right\|_2\\
    &\leq \left\|\begin{bmatrix}
    (1+\beta_k)I-\alpha_kH_k & -\beta_k I\\
    I & 0
    \end{bmatrix}\right\|_2\left\|\begin{bmatrix}
    \theta_k-\theta^*\\
    \theta_{k-1}-\theta^*
    \end{bmatrix}\right\|_2
\end{align*}
where $H_k = (1 - \cos\phi^k_{12}/R_k)\nabla^2\mathcal{L}_1(z_k) + (1 - \cos\phi^k_{12} \cdot R_k)\nabla^2\mathcal{L}_2(z'_k)$.

By strong convexity and smoothness of $\loss_1$ and $\loss_2$, we have the eigenvalues of $\nabla^2\mathcal{L}_1(z_k)$ are between $\mu_1$ and $L_1$. Similarly, the eigenvalues of $\nabla^2\mathcal{L}_2(z'_k)$ are between $\mu_2$ and $L_2$. Thus the eigenvalues of $H_k$ are between $\mu_k = (1 - \cos\phi^k_{12}/R_k)\mu_1 + (1 - \cos\phi^k_{12}\cdot R_k)\mu_2$ and $L_k = (1 - \cos\phi^k_{12}/R_k)L_1 + (1 - \cos\phi^k_{12}\cdot R_k)L_2$~\cite{fulton2000eigenvalues}. Hence following Lemma 3.1 in ~\cite{notes}, we have $$\left\|\begin{bmatrix}
    (1+\beta_k)I-\alpha_kH_k & -\beta_k I\\
    I & 0
    \end{bmatrix}\right\|_2 \leq \max\{|1 - \sqrt{\alpha_k\mu_k}|,|1 - \sqrt{\alpha_kL_k}|\}.$$
Thus we have
\begin{align}
    \left\|\begin{bmatrix}
    \theta_{k+1} - \theta^*\\
    \theta_k - \theta^*
    \end{bmatrix}\right\|_2 &\leq \max\{|1 - \sqrt{\alpha_k\mu_k}|,|1 - \sqrt{\alpha_kL_k}|\}\left\|\begin{bmatrix}
    \theta_k-\theta^*\\
    \theta_{k-1}-\theta^*
    \end{bmatrix}\right\|_2\nonumber\\
    &= \frac{\sqrt{\kappa_k}-1}{\sqrt{\kappa_k}+1}\left\|\begin{bmatrix}
    \theta_k-\theta^*\\
    \theta_{k-1}-\theta^*
    \end{bmatrix}\right\|_2\label{eq:sub}\\
    &\leq \left\|\begin{bmatrix}
    \theta_k-\theta^*\\
    \theta_{k-1}-\theta^*
    \end{bmatrix}\right\|_2\nonumber
\end{align}
where $\kappa_k = \frac{L_k}{\mu_k}$ and Equation~\ref{eq:sub} follows from substitution $\alpha_k = \frac{4}{\sqrt{L_k} + \sqrt{\mu_k}}$. Hence PCGrad with heavy ball method converges linearly if $\cos\phi^k_{12} \neq -1$.
\end{proof}


%
%

%
%
%
%
%
%
%
%

%
%
%
%
%
%
%
%
%

\section{Empirical Objective-Wise Evaluations of PCGrad}
\label{app:objective-wise}

In this section, we visualize the per-task training loss and validation loss curves respectively on NYUv2. 
%
The goal of measuring objective-wise performance is to study the convergence of PCGrad in practice, particularly amidst the possibility of slow convergence due to cosine similarities near -1, as discussed in Section~\ref{sec:theory}.
%

We show the objective-wise evaluation results 
%
on NYUv2 in Figure~\ref{fig:nyuv2_curve}. 
%
%
For evaluations on NYUv2, PCGrad + MTAN attains similar training convergence rate compared to MTAN in three tasks in NYUv2 while converging faster and achieving lower validation loss in 2 out of 3 tasks. Note that in task 0 of the NYUv2 dataset, both methods seem to overfit, suggesting a better regularization scheme for this domain.

In general, these results suggest that PCGrad has a regularization effect on supervised multi-task learning, rather than an improvement on optimization speed or convergence. We hypothesize that this regularization is caused by PCGrad leading to greater sharing of representations across tasks, such that the supervision for one task better regularizes the training of another. 
This regularization effect seems notably different from the effect of PCGrad on reinforcement learning problems, where PCGrad dramatically improves training performance. This suggests that multi-task supervised learning and multi-task reinforcement learning problems may have distinct challenges.

\begin{figure*}[ht]
    \centering
    \includegraphics[width=0.325\columnwidth]{figures/nyuv2_train_task0.png}
    \includegraphics[width=0.325\columnwidth]{figures/nyuv2_train_task1.png}
    \includegraphics[width=0.325\columnwidth]{figures/nyuv2_train_task2.png}
    \includegraphics[width=0.325\columnwidth]{figures/nyuv2_val_task0.png}
    \includegraphics[width=0.325\columnwidth]{figures/nyuv2_val_task1.png}
    \includegraphics[width=0.325\columnwidth]{figures/nyuv2_val_task2.png}
    \vspace{-0.2cm}
    \caption{\footnotesize Empirical objective-wise evaluations on NYUv2. On the top row, we show the objective-wise training learning curves and on the bottom row, we show the objective-wise validation learning curves. PCGrad+MTAN converges with a similar rate compared to MTAN in training and for validation losses, PCGrad+MTAN converges faster and obtains a lower final validation loss in two out of three tasks. This result corroborate that in practice, PCGrad does not exhibit the potential slow convergence problem shown in Theorem~\ref{thm:converge}.}
    \vspace{-0.1cm}
    \label{fig:nyuv2_curve}
\end{figure*}


\section{Practical Details of PCGrad on Multi-Task and Goal-Conditioned Reinforcement Learning}
\label{app:practical_rl}

\neurips{In our experiments, we apply PCGrad to the soft actor-critic (SAC) algorithm~\citep{haarnoja2018sac}, an off-policy RL method.} In SAC, we employ a Q-learning style gradient to compute the gradient of the Q-function network, $Q_{\phi}(s,a, z_i)$, often known as the critic, and a reparameterization-style gradient to compute the gradient of the policy network $\pi_{\theta}(a|s, z_i)$, often known as the actor. For sampling, we instantiate a set of replay buffers $\{\data_{i}\}_{\task_i \sim p(\task)}$. Training and data collection are alternated throughout training. During a data collection step, we run the policy $\pi_\theta$ on all the tasks $\task_i \sim p(\task)$ to collect an equal number of paths for each task and store the paths of each task $\task_i$ into the corresponding replay buffer $\data_i$. At each training step, we sample an equal amount of data from each replay buffer $\data_i$ to form a stratified batch. For each task $\task_i \sim p(\task)$, the parameters of the critic $\theta$ are optimized to minimize the soft Bellman residual:
\begin{align}
    J^{(i)}_Q(\phi) &= \mathbb{E}_{(s_t, a_t, z_i) \sim \mathcal{D}_{i}}\left[Q_\phi(s_t, a_t, z_i) - (r(s_t, a_t, z_i)+ \gamma V_{\bar{\phi}}(s_{t+1}, z_i))\right]\text{,}
\end{align}
\begin{align}
V_{\bar{\phi}}(s_{t+1}, z_i) &= \mathbb{E}_{a_{t+1}\sim\pi_\theta}\left[Q_{\bar{\phi}}(s_{t+1}, a_{t+1}, z_i)- \alpha\log\pi_\theta(a_{t+1}|s_{t+1}, z_i)\right]\text{,}
\end{align}
where $\gamma$ is the discount factor, $\bar{\phi}$ are the delayed parameters, and $\alpha$ is a learnable temperature that automatically adjusts the weight of the entropy term. For each task $\task_i \sim p(\task)$, the parameters of the policy $\pi_\theta$ are trained to minimize the following objective
\begin{align}
    J^{(i)}_\pi(\theta) &= \mathbb{E}_{s_t \sim \mathcal{D}_{i}}\left[\mathbb{E}_{a_t \sim \pi_\theta(a_t|s_t, z_i))} \left[\alpha\log\pi_\theta(a_{t}|s_{t}, z_i)- Q_\phi(s_t, a_t, z_i)\right]\right]\text{.}
\end{align}
%
We compute $\nabla_\phi J^{(i)}_Q(\phi)$ and $\nabla_\theta J^{(i)}_\pi(\theta)$ for all $\task_i \sim p(\task)$ and apply PCGrad to both following Algorithm~\ref{alg:dgrad-o}.

In the context of SAC specifically, we also propose to learn the temperature $\alpha$ for adjusting entropy of the policy on a per-task basis. This allows the method to control the entropy of the multi-task policy per-task. The motivation is that if we use a single learnable temperature for adjusting entropy of the multi-task policy $\pi_\theta(a|s, z_i)$, SAC may stop exploring once all easier tasks are solved, leading to poor performance on tasks that are harder or require more exploration. To address this issue, we propose to learn the temperature on a per-task basis as mentioned in Section~\ref{sec:pcgrad_practical}, i.e. using a parametrized model to represent $\alpha_\psi(z_i)$. This allows the method to control the entropy of $\pi_\theta(a|s, z_i)$ per-task.
%
We optimize the parameters of $\alpha_\psi(z_i)$ using the same constrained optimization framework as in~\cite{haarnoja2018sac}.

When applying PCGrad to goal-conditioned RL, we represent $p(\task)$ as a distribution of goals and let $z_i$ be the encoding of a goal.
Similar to the multi-task supervised learning setting discussed in Section~\ref{sec:pcgrad_practical}, PCGrad may be combined with various architectures designed for multi-task and goal-conditioned RL~\citep{fernando2017pathnet, devin2016modularnet}, where PCGrad operates on the gradients of shared parameters, leaving task-specific parameters untouched.



\section{2D Optimization Landscape Details}
To produce the 2D optimization visualizations in Figure \ref{fig:optlandscape}, we used a parameter vector $\theta = [ \theta_1, \theta_2 ] \in \mathbbm{R}^2$ and the following task loss functions:
\begin{align*}
&\loss_1(\theta) = 20 \log(\max(|.5\theta_1 + \tanh(\theta_2)|, 0.000005)) \\
&\loss_2(\theta) = 25 \log(\max(|.5\theta_1  - \tanh(\theta_2) + 2| , 0.000005))
\end{align*}
The multi-task objective is $\loss(\theta) = \loss_1(\theta) + \loss_2(\theta)$. We initialized $\theta = [0.5, -3]$ and performed 500,000 gradient updates to minimize $\loss$ using the Adam optimizer with learning rate $0.001$. We compared using Adam for each update to using Adam in conjunction with the PCGrad method presented in Section \ref{sec:pcgrad}.
\label{app:optimization_landscape_details}

\section{Additional Multi-Task Supervised Learning Results}
\label{app:additional_mt_sup_results}

We present our multi-task supervised learning results on MultiMNIST and CityScapes here.
\paragraph{MultiMNIST.} \arxiv{Following the same set-up of~\citet{sener2018multi}, for each image, we sample a different one uniformly at random. Then we put one of the image on the top left and the other one on the bottom right. The two tasks in the multi-task learning problem are to classify the digits on the top left (task-L) and bottom right (task-R) respectively. We construct such $60$K examples. We combine PCGrad with the same backbone architecture used in~\citep{sener2018multi} and compare its performance to~\citet{sener2018multi} by running the open-sourced code provided in~\citep{sener2018multi}. As shown in Table~\ref{tbl:multimnist}, PCGrad results 0.13\% and 0.55\% improvement over~\cite{sener2018multi} in left and right digit accuracy respectively.}

\begin{table}[h]
    \begin{center}
    %
    %
    \begin{small}
    \begin{tabular}{l|c|c}
    \toprule
        & left digit &right digit  \\
      \midrule
      \citet{sener2018multi} & 96.45 & 94.95\\
      \midrule
      PCGrad (ours) & \bf 96.58 & \bf 95.50 \\
      \bottomrule
    \end{tabular}
    \end{small}
    \end{center}
    \caption{\footnotesize \arxiv{MultiMNIST results. PCGrad achieves improvements over the approach by~\citet{sener2018multi} in both left and right digit classfication accuracy.}}
    \vspace{-0.4cm}
    \label{tbl:multimnist}
\end{table}

\paragraph{CityScapes.}

The CityScapes dataset~\cite{cordts2016cityscapes} contains 19 classes of street-view images resized to $128 \times 256$. There are two tasks in this dataset: semantic segmentation and depth estimation. Following the setup in \citet{liu2018attention}, we pair the depth estimation task with semantic segmentation using the coarser 7 categories instead of the finer 19 classes in the original CityScapes dataset. Similar to NYUv2 evaluations described in Section~\ref{sec:experiments}, we also combine PCGrad with MTAN~\cite{liu2018attention} and compare it to a range of methods discussed in Appendix~\ref{app:sl_details}. For the combination of PCGrad and MTAN, we only use equal weighting as discussed in~\cite{liu2018attention} as we find it working well in practice. We present the results in Table~\ref{tbl:cityscapes_partial}. As shown in Table~\ref{tbl:cityscapes_partial}, PCGrad + MTAN outperforms MTAN in three out of four scores while obtaining the top scores in both mIoU abd pixel accuracy for the semantic segmentation task, suggesting the effectiveness of PCGrad on realistic image datasets. We also provide the full results including three different weighting schemes in Table~\ref{tbl:cityscapes} in Appendix~\ref{app:full_nyuv2_results}.

\begin{table}[h]
  \centering
  \small
  \def\arraystretch{0.9}
  \setlength{\tabcolsep}{0.35em}
  \begin{tabularx}{0.6\linewidth}{cll*{4}{c}}
  \toprule
 \multicolumn{1}{c}{\multirow{2.5}[4]{*}{\#P.}} & \multicolumn{1}{c}{\multirow{2.5}[4]{*}{Architecture}} & \multicolumn{2}{c}{Segmentation} & \multicolumn{2}{c}{Depth}  \\
  \cmidrule(lr){3-4} \cmidrule(lr){5-6}
   &\multicolumn{1}{c}{} & \multicolumn{2}{c}{(Higher Better)} & \multicolumn{2}{c}{(Lower Better)} \\
     \multicolumn{1}{c}{} & \multicolumn{1}{c}{} & \multicolumn{1}{c}{mIoU}  & \multicolumn{1}{c}{Pix Acc}  & \multicolumn{1}{c}{Abs Err} & \multicolumn{1}{c}{Rel Err}  \\
\midrule
   2 &   One Task & 51.09 & 90.69 & 0.0158 & 34.17   \\
   3.04 & STAN   &  51.90&90.87 & 0.0145 & \mybox{\bf 27.46} \\
  \midrule
    1.75 & Split, Wide  & 50.17 & 90.63  & 0.0167  & 44.73  \\
   \cmidrule(lr){1-6}
    2& Split, Deep &  49.85  & 88.69 & 0.0180  & 43.86   \\
   \cmidrule(lr){1-6}
    3.63 & Dense &  51.91  & 90.89  & \mybox{\bf 0.0138} & 27.21  \\
  \cmidrule(lr){1-6}
    $\approx$2& Cross-Stitch \cite{misra2016cross} &  50.08 & 90.33 & 0.0154 & 34.49    \\
  \cmidrule(lr){1-6}
    1.65 & MTAN   & 53.04 & 91.11 &  0.0144 & 33.63  \\
  \cmidrule(lr){1-6}
     1.65& PCGrad+MTAN (Ours) & \mybox{\bf 53.59} & \mybox{\bf 91.45} & 0.0171 & 31.34\\
    \bottomrule
    \end{tabularx}%
     \caption{We present the 7-class semantic segmentation and depth estimation results on CityScapes dataset. We use \#P to denote the number of parameters of the network. We use box and bold text to highlight the method that achieves the best validation score for each task. As seen in the results, PCGrad+MTAN with equal weights outperforms MTAN with equal weights in three out of four scores while achieving the top score both scores in the segmentation task.}
    \label{tbl:cityscapes_partial}
    \vspace{-0.25cm}
\end{table}

%

%

%

%
%
%
%
%
%
%
%
%
%
%
%
%
%
%
%
%
%
%
%

\section{Goal-Conditioned Reinforcement Learning Results}
\label{app:goal-conditioned}
%

\neurips{For our goal-conditioned RL evaluation, \icml{we adopt the goal-conditioned robotic pushing task with a Sawyer robot} where the goals are represented as the concatenations of the initial positions of the puck to be pushed and the its goal location, both of which are uniformly sampled (details in Appendix~\ref{app:goal_conditioned_details}). We also apply the temperature adjustment strategy as discussed in Section~\ref{sec:pcgrad_practical} to predict the temperature for entropy term given the goal. We summarize the results in Figure~\ref{fig:goal-conditioned}. PCGrad with SAC achieves better performance in terms of average distance to the goal position, while the vanilla SAC agent is struggling to successfully accomplish the task. This suggests that PCGrad is able to ease the RL optimization problem also when the task distribution is continuous.}

\begin{figure*}[ht]
    \centering
    \includegraphics[width=0.7\columnwidth]{figures/PCGrad_push_random_goals_init_hardest_itr600_3_random_seed_nopa_legend.png}
    \vspace{-0.2cm}
    \caption{\footnotesize We present the goal-conditioned RL results. PCGrad outperforms vanilla SAC in terms of both average distance the goal and data efficiency.}
    \vspace{-0.1cm}
    \label{fig:goal-conditioned}
\end{figure*}


%
%
%

%

%

%
%
%
%
%
%
%
%

%
%
%
%
%
%
%
%



\section{Comparison to CosReg}
\label{app:cosine}

\icmllast{We compare PCGrad to a prior method CosReg~\citep{suteu2019regularizing}, which adds a regularization term to force the cosine similarity between gradients of two different tasks to stay $0$. PCGrad achieves much better average success rate in MT10 benchmark as shown in Figure~\ref{fig:cosreg}. Hence, while it's important to reduce interference between tasks, it's also crucial to keep the task gradients that enjoy positive cosine similarities in order to ensure sharing across tasks.}

\begin{figure*}[ht]
    \centering
    \includegraphics[width=0.5\columnwidth]{figures/PCGrad_mt10_itr900_pcgrad_ablation_nopa_cosine.png}
    \vspace{-0.2cm}
    \caption{\footnotesize Comparison between PCGrad and CosReg~\citep{suteu2019regularizing}. PCGrad outperforms CosReg, suggesting that we should both reduce the interference and keep shared structure across tasks.}
    \vspace{-0.1cm}
    \label{fig:cosreg}
\end{figure*}

\section{Ablation study on the task order}
\label{app:ablation_order}

\arxiv{As stated on line 4 in Algorithm~\ref{alg:dgrad-o}, we sample the tasks from the batch and randomly shuffle the order of the tasks before performing the update steps in PCGrad. With random shuffling, we make PCGrad symmetric w.r.t. the task order in expectation. In Figure~\ref{fig:task_order}, we observe that PCGrad with a random task order achieves better performance between PCGrad with a fixed task order in the setting of MT50 where the number of tasks is large and the conflicting gradient phenomenon is much more likely to happen.}

\begin{figure}[h]
    \centering
    \includegraphics[width=0.6\columnwidth]{figures/PCGrad_mt50_itr400_random_order_nopa_ablation.png}
    \caption{\footnotesize \arxiv{Ablation study on using a fixed task order during PCGrad. PCGrad with a random task order does significantly better PCGrad with a fixed task order in MT50 benchmark.}}
    \vspace{-0.2cm}
    \label{fig:task_order}
\end{figure}

\section{Combining PCGrad with other architectures}
\label{app:combined}

In this subsection, we test whether PCGrad can improve performances when combined with more methods. In Table~\ref{tab:nyuv2_combined}, we find that PCGrad does improve the performance in all four metrics of the three tasks on the NYUv2 dataset when combined with Cross-Stitch~\citep{misra2016cross} and Dense. In Figure~\ref{fig:mt10_combined}, we also show that PCGrad + Multi-head SAC outperforms Multi-head SAC on its own. These results suggest that PCGrad can be flexibly combined with any multi-task learning architectures to further improve performance.

\begin{table}[h]
  \def\arraystretch{0.9}
  \setlength{\tabcolsep}{0.42em}
    \begin{tabularx}{0.9\linewidth}{cccc*{9}{c}}
  \toprule
 \multicolumn{1}{c}{\multirow{2}[2]{*}{Method}} & \multicolumn{1}{c}{Segmentation} & \multicolumn{1}{c}{Depth}  & \multicolumn{2}{c}{Surface Normal}\\
  \cmidrule(lr){4-5}
   \multicolumn{1}{c}{} & \multicolumn{1}{c}{mIoU} &  \multicolumn{1}{c}{Abs Err}   & \multicolumn{1}{c}{Angle Distance}  & \multicolumn{1}{c}{Within $11.25^\circ$} \\
\midrule
  Cross-Stitch   & 15.69  &  0.6277   & 32.69 &  21.63 \\
  Cross-Stitch + PCGrad  &  {\bf 18.14}  &  {\bf 0.5805}   & {\bf 31.38} &  {\bf 21.75}  \\
  \cmidrule(lr){1-5}
  Dense & 16.48  & 0.6282  & 31.68 & 21.73\\
  Dense + PCGrad & {\bf 18.08}  &  {\bf 0.5850}   & {\bf 30.17} &  {\bf 23.29}\\
%
%
%
    \bottomrule
    \end{tabularx}
    \vspace{0.1cm}
         \caption{\footnotesize We show the performances of PCGrad combined with other methods on three-task learning on the NYUv2 dataset, where PCGrad further improves the results of prior multi-task learning architectures.
     \label{tab:nyuv2_combined}
    %
     }
\end{table}

\begin{figure}[h]
		\centering
		\includegraphics[width=0.6\columnwidth]{figures/PCGrad_mt10_itr900_multihead_pcgrad.png}
		\vspace{-0.1cm}
		\captionof{figure}{\footnotesize We show the comparison between Multi-head SAC and Multi-head SAC + PCGrad on MT10. Multi-head SAC + PCGrad outperforms Multi-head SAC, suggesting that PCGrad can improves the performance of multi-headed architectures in the multi-task RL settings.}
		\label{fig:mt10_combined}
%
\end{figure}

\section{Experiment Details}
\label{app:exp_details}

\subsection{Multi-Task Supervised Learning Experiment Details}
\label{app:sl_details}

For all the multi-task supervised learning experiments, PCGrad converges within 12 hours on a NVIDIA TITAN RTX GPU while the vanilla models without PCGrad converge within 8 hours. PCGrad consumes at most 10 GB memory on GPU while the vanilla method consumes 6GB on GPU among all experiments.

For our CIFAR-100 multi-task experiment, we adopt the architecture used in~\cite{rosenbaum2019routing}, which is a convolutional neural network that consists of 3 convolutional layers with $160$ $3\times3$ filters each layer and 2 fully connected layers with $320$ hidden units. As for experiments on the NYUv2 dataset, we follow~\cite{liu2018attention} to use SegNet~\citep{badrinarayanan2017segnet} as the backbone architecture.

We use five algorithms as baselines in the CIFAR-100 multi-task experiment: \textbf{task specific-1-fc} \citep{rosenbaum2017routing}: a convolutional neural network shared across tasks except that each task has a separate last fully-connected layer, \textbf{task specific-1-fc} \citep{rosenbaum2017routing} : all the convolutional layers shared across tasks with separate fully-connected layers for each task, \textbf{cross stitch-all-fc} \citep{misra2016crossstitch}: one convolutional neural network per task along with cross-stitch units to share features across tasks, \textbf{routing-all-fc + WPL} \citep{rosenbaum2019routing}: a network that employs a trainable router trained with multi-agent RL algorithm (WPL) to select trainable functions for each task, \textbf{independent}: training separate neural networks for each task.
%

For comparisons on the NYUv2 dataset, we consider 5 baselines: \textbf{Single Task, One Task}: the vanilla SegNet used for single-task training, \textbf{Single Task, STAN}~\citep{liu2018attention}: the single-task version of MTAN as mentioned below, \textbf{Multi-Task, Split, Wide / Deep}~\citep{liu2018attention}: the standard SegNet shared for all three tasks except that each task has a separate last layer for final task-specific prediction with two variants \textbf{Wide} and \textbf{Deep} specified in ~\cite{liu2018attention}, \textbf{Multi-Task Dense}: a shared network followed by separate task-specific networks, \textbf{Multi-Task Cross-Stitch}~\citep{misra2016crossstitch}: similar to the baseline used in CIFAR-100 experiment but with SegNet as the backbone, \textbf{MTAN}~\citep{liu2018attention}: a shared network with a soft-attention module for each task.

\begin{figure}[t]
    \centering
    \includegraphics[width=1.0\columnwidth]{figures/MT50.png}
    \vspace{-0.7cm}
    \caption{\footnotesize The $50$ tasks of MT50 from Meta-World~\citep{metaworld}. MT10 is a subset of these tasks, which includes reach, push, pick \& place, open drawer, close drawer, open door, press button top, open window, close window, and insert peg inside.
    }
    \label{fig:metaworld}
\end{figure}

\subsection{Multi-Task Reinforcement Learning Experiment Details}
\label{app:rl_details}

Our reinforcement learning experiments all use the SAC~\citep{haarnoja2018sac} algorithm as the base algorithm, where the actor and the critic are represented as 6-layer fully-connected feedforward neural networks for all methods. The numbers of hidden units of each layer of the neural networks are $160$, $300$ and $200$ for MT10, MT50 and goal-conditioned RL respectively.  For the multi-task RL experiments, PCGrad + SAC converges in 1 day (5M simulation steps) and 5 days (20M simulation steps) on the MT10 and MT50 benchmarks respectively on a NVIDIA TITAN RTX GPU while vanilla SAC converges in 12 hours and 3 days on the two benchmarks respectively. PCGrad + SAC consumes 1 GB and 6 GB memory on GPU on the MT10 and MT50 benchmarks respectively while the vanilla SAC consumes 0.5 GB and 3 GB respectively.

\icml{In the case of multi-task reinforcement learning, we evaluate our algorithm on the recently proposed Meta-World benchmark~\citep{metaworld}. This benchmark includes a variety of simulated robotic manipulation tasks contained in a shared, table-top environment with a simulated Sawyer arm (visualized in Fig.~\ref{fig:metaworld}). 
In particular, we use the multi-task \arxiv{benchmarks MT10 and MT50}, which consists of the 10 tasks \arxiv{and 50 tasks respectively} depicted in Fig.~\ref{fig:metaworld} that require diverse strategies to solve them, which makes them difficult to optimize jointly with a single policy. \arxiv{Note that MT10 is a subset of MT50.}}
%
At each data collection step, we collect $600$ samples for each task, and at each training step, we sample $128$ datapoints per task from corresponding replay buffers. We measure success according to the metrics used in the Meta-World benchmark where the reported the success rates are averaged across tasks. For all methods, we apply the temperature adjustment strategy as discussed in Section~\ref{sec:pcgrad_practical} to learn a separate alpha term per task as the task encoding in MT10 and MT50 is just a one-hot encoding. 

On the multi-task and goal-conditioned RL domain, we apply PCGrad to the vanilla SAC algorithm with task encoding as part of the input to the actor and the critic as described in Section~\ref{sec:pcgrad_practical} and compare PCGrad to the vanilla \textbf{SAC} without PCGrad and training actors and critics for each task individually (\textbf{Independent}).

\subsection{Goal-conditioned Experiment Details}
\label{app:goal_conditioned_details}

We use the pushing environment from the Meta-World benchmark~\citep{metaworld} as shown in Figure~\ref{fig:metaworld}. In this environment, the table spans from $[-0.4, 0.2]$ to $[0.4, 1.0]$ in the 2D space. To construct the goals, we sample the intial positions of the puck from the range $[-0.2, 0.6]$ to $[0.2, 0.7]$ on the table and the goal positions from the range $[-0.2, 0.85]$ to $[0.2, 0.95]$ on the table. The goal is represented as a concatenation of the initial puck position and the goal position. Since in the goal-conditioned setting, the task distribution is continuous, we sample a minibatch of $9$ goals and $128$ samples per goal at each training iteration and also sample $600$ samples per goal in the minibatch at each data collection step.


\subsection{Full CityScapes and NYUv2 Results}
\label{app:full_nyuv2_results}

We provide the full comparison on the CityScapes and NYUv2 datasets in Table~\ref{tbl:cityscapes} and  Table~\ref{tab:nyu_results_full} respectively.

\begin{table}[h]
  \centering
  \small
  \def\arraystretch{0.9}
  \setlength{\tabcolsep}{0.35em}
  \begin{tabularx}{0.82\linewidth}{cll*{4}{c}}
  \toprule
 \multicolumn{1}{c}{\multirow{2.5}[4]{*}{\#P.}} & \multicolumn{1}{c}{\multirow{2.5}[4]{*}{Architecture}} & \multicolumn{1}{c}{\multirow{2.5}[4]{*}{Weighting}} & \multicolumn{2}{c}{Segmentation} & \multicolumn{2}{c}{Depth}  \\
  \cmidrule(lr){4-5} \cmidrule(lr){6-7}
   &\multicolumn{1}{c}{} & \multicolumn{1}{c}{} & \multicolumn{2}{c}{(Higher Better)} & \multicolumn{2}{c}{(Lower Better)} \\
     \multicolumn{1}{c}{} & \multicolumn{1}{c}{} & \multicolumn{1}{c}{} & \multicolumn{1}{c}{mIoU}  & \multicolumn{1}{c}{Pix Acc}  & \multicolumn{1}{c}{Abs Err} & \multicolumn{1}{c}{Rel Err}  \\
\midrule
   2 &   One Task  &n.a. & 51.09 & 90.69 & 0.0158 & 34.17   \\
   3.04 & STAN & n.a.   &  51.90&90.87 & 0.0145 & 27.46 \\
  \midrule
    &   &Equal Weights  & 50.17 & 90.63  & 0.0167  & 44.73  \\
   1.75 & Split, Wide    & Uncert. Weights \cite{kendall2017multi}    & {\bf 51.21} & {\bf 90.72} & {\bf 0.0158} &  44.01   \\
    &    & DWA, $T = 2$ & 50.39 & 90.45 & 0.0164 & {\bf 43.93}     \\
   \cmidrule(lr){1-7}
    &   &Equal Weights  &  {\bf 49.85}  & 88.69 & 0.0180  & 43.86   \\
    2  & Split, Deep    & Uncert. Weights \cite{kendall2017multi}   & 48.12  & 88.68 & {\bf 0.0169} & {\bf 39.73}  \\
    &    & DWA, $T = 2$ & 49.67  & {\bf 88.81}& 0.0182 & 46.63   \\
   \cmidrule(lr){1-7}
    &    & Equal Weights & {\bf 51.91}  & 90.89  & 0.0138 & 27.21  \\
   3.63 & Dense   &Uncert. Weights \cite{kendall2017multi}  &51.89 & {\bf 91.22} & \mybox{\bf 0.0134} & \mybox{\bf 25.36}    \\
    &     &DWA, $T = 2$   & 51.78 & 90.88 & 0.0137 & 26.67     \\
  \cmidrule(lr){1-7}
    &    & Equal Weights &  50.08 & 90.33 & 0.0154 & 34.49    \\
   $\approx$2& Cross-Stitch \cite{misra2016cross}    &Uncert. Weights \cite{kendall2017multi}  & 50.31 & 90.43 & {\bf 0.0152} & {\bf 31.36}    \\
   &     &DWA, $T = 2$   & {\bf 50.33} & {\bf 90.55} & 0.0153 & 33.37 \\
  \cmidrule(lr){1-7}
    &    & Equal Weights   & 53.04 & 91.11 &  {\bf 0.0144} & 33.63  \\
  1.65 & MTAN & Uncert. Weights \cite{kendall2017multi} & \mybox{\bf 53.86} & 91.10  &0.0144  & 35.72  \\
      & &DWA, $T = 2$     & 53.29 & 91.09 & 0.0144  & 34.14 \\
     1.65& PCGrad+MTAN (Ours) & Equal Weights & 53.59 & \mybox{\bf 91.45} & 0.0171 & {\bf 31.34}\\
    \bottomrule
    \end{tabularx}%
     \caption{We present the 7-class semantic segmentation and depth estimation results on CityScapes dataset. We use \#P to denote the number of parameters of the network, and the best performing variant of each architecture is highlighted in bold. We use box to highlight the method that achieves the best validation score for each task. As seen in the results, PCGrad+MTAN with equal weights outperforms MTAN with equal weights in three out of four scores while achieving the top score in pixel accuracy across all methods.}
    \label{tbl:cityscapes}
    \vspace{-0.25cm}
\end{table}

\begin{table*}[h]
  \centering
  \scriptsize
  \def\arraystretch{0.9}
%
  \setlength{\tabcolsep}{0.2em}
%
  \begin{tabularx}{0.99\linewidth}{ccll*{9}{c}}
  \toprule
  \multicolumn{1}{c}{\multirow{3.5}[4]{*}{Type}} & \multicolumn{1}{c}{\multirow{3.5}[4]{*}{\#P.}} & \multicolumn{1}{c}{\multirow{3.5}[4]{*}{Architecture}} & \multicolumn{1}{c}{\multirow{3.5}[4]{*}{Weighting}} & \multicolumn{2}{c}{Segmentation} & \multicolumn{2}{c}{Depth}  & \multicolumn{5}{c}{Surface Normal}\\
  \cmidrule(lr){5-6} \cmidrule(lr){7-8} \cmidrule(lr){9-13}
   &\multicolumn{1}{c}{}  &\multicolumn{1}{c}{} & \multicolumn{1}{c}{} & \multicolumn{2}{c}{\multirow{1.5}[2]{*}{(Higher Better)}} & \multicolumn{2}{c}{\multirow{1.5}[2]{*}{(Lower Better)}}   & \multicolumn{2}{c}{Angle Distance}  & \multicolumn{3}{c}{Within $t^\circ$} \\
    &\multicolumn{1}{c}{}  &\multicolumn{1}{c}{} & \multicolumn{1}{c}{} & \multicolumn{1}{c}{} & \multicolumn{1}{c}{}  & \multicolumn{1}{c}{}  & \multicolumn{1}{c}{} & \multicolumn{2}{c}{(Lower Better)} & \multicolumn{3}{c}{(Higher Better)} \\
     &\multicolumn{1}{c}{} & \multicolumn{1}{c}{} & \multicolumn{1}{c}{} & \multicolumn{1}{c}{mIoU}  & \multicolumn{1}{c}{Pix Acc}  & \multicolumn{1}{c}{Abs Err} & \multicolumn{1}{c}{Rel Err} & \multicolumn{1}{c}{Mean}  & \multicolumn{1}{c}{Median}  & \multicolumn{1}{c}{11.25} & \multicolumn{1}{c}{22.5} & \multicolumn{1}{c}{30} \\
\midrule
  \multirow{2}*{Single Task}  &  3 &   One Task  &n.a.  &  15.10  &51.54  & 0.7508 & 0.3266      &  31.76& 25.51 & 22.12 & 45.33 &  57.13\\
   & 4.56 & STAN$^\dagger$ & n.a.    &  15.73 & 52.89 &  0.6935 &  0.2891 & 32.09 & 26.32 & 21.49 &44.38  & 56.51\\
  \midrule
  \multirow{15}*{Multi Task}  &  &   &Equal Weights & 15.89  & 51.19   &  0.6494   & 0.2804 & 33.69 & 28.91 & {18.54} & 39.91 & 52.02 \\
  & 1.75 & Split, Wide    & Uncert. Weights$^*$  & 15.86  & 51.12  &  {\bf 0.6040} & 0.2570 & {\bf 32.33}  & {\bf 26.62} & {\bf 21.68} & {\bf 43.59} & {\bf 55.36 }\\
  &  &    & DWA$^\dagger$, $T = 2$  & {\bf 16.92}  & {\bf 53.72}  &  0.6125  & {\bf 0.2546} & 32.34  &27.10  & 20.69  &  42.73 & 54.74 \\
   \cmidrule(lr){2-13}
    &  &   &Equal Weights & 13.03  & 41.47  &  0.7836 & 0.3326 & 38.28  & 36.55 & 9.50 & 27.11 & 39.63 \\
  & 2 & Split, Deep    & Uncert. Weights$^*$  & {\bf 14.53}  & 43.69  & 0.7705  & 0.3340 & {\bf 35.14}  & {\bf 32.13 }& {\bf 14.69} & {\bf 34.52} &  {\bf 46.94 }\\
  &  &    & DWA$^\dagger$, $T = 2$  & 13.63  & {\bf 44.41}  & {\bf 0.7581}  & {\bf 0.3227 }& 36.41  & 34.12 & 12.82 & 31.12 & 43.48 \\
   \cmidrule(lr){2-13}
  &  &    &Equal Weights  & 16.06   & 52.73   &  0.6488   & 0.2871  & 33.58 & 28.01 & {20.07} & 41.50 & 53.35 \\
  & 4.95 & Dense   &Uncert. Weights$^*$  & {\bf 16.48}  & {\bf 54.40}  &   0.6282  & 0.2761  & {\bf 31.68} & {\bf 25.68}  & {\bf 21.73}  & {\bf 44.58} & {\bf 56.65} \\
  &  &     &DWA$^\dagger$, $T = 2$  &  16.15  & 54.35   &  {\bf 0.6059}  &  {\bf 0.2593} &  32.44 & 27.40 & 20.53  & 42.76 & 54.27\\
  \cmidrule(lr){2-13}
  &  &    &Equal Weights   & 14.71  & 50.23   &  0.6481   &  0.2871 & 33.56 & 28.58 & 20.08 & 40.54 & 51.97\\
  &  $\approx$3 & Cross-Stitch$^\ddagger$    &Uncert. Weights$^*$  &  15.69 &  52.60  &  0.6277   & 0.2702 & 32.69 &  27.26 & 21.63 &  42.84 &  54.45 \\
  &  &     &DWA$^\dagger$, $T = 2$   & {\bf 16.11} &  {\bf  53.19} & {\bf 0.5922}   & {\bf 0.2611} & {\bf 32.34} & {\bf 26.91} & {\bf 21.81} & {\bf 43.14} & {\bf 54.92} \\
  \cmidrule(lr){2-13}
  &  &    & Equal Weights &   {\bf 17.72} &  55.32    & {\bf 0.5906}  &  0.2577 & 31.44  & {\bf 25.37}&  \mybox{\bf 23.17} & 45.65 &57.48\\
  &1.77 & MTAN$^\dagger$ & Uncert. Weights$^*$  &  17.67    &  {\bf 55.61}  &  0.5927    & 0.2592 & {\bf 31.25} & 25.57 & 22.99 & {\bf 45.83} & {\bf 57.67} \\
    &    & &DWA$^\dagger$, $T = 2$   &  17.15    &  54.97 &  0.5956  & {\bf 0.2569} & 31.60 & 25.46 & 22.48 & 44.86 & 57.24 \\
      \cmidrule(lr){2-13}
  &1.77 & MTAN$^\dagger$ + PCGrad (ours) & Uncert. Weights$^*$  &  \mybox{\bf 20.17}    &  \mybox{\bf 56.65}  &  \mybox{\bf 0.5904}    & \mybox{\bf 0.2467} & \mybox{\bf 30.01} & \mybox{\bf 24.83} & 22.28 & \mybox{\bf 46.12} & \mybox{\bf 58.77} \\
    \bottomrule
    \end{tabularx}
     \caption{We present the full results on three tasks on the NYUv2 dataset: 13-class semantic segmentation, depth estimation, and surface normal prediction results. \#P shows the total number of network parameters. We highlight the best performing combination of multi-task architecture and weighting in bold. The top validation scores for each task are annotated with boxes. The symbols indicate prior methods: $^*$: \citep{kendall2017multi}, $^\dagger$: \citep{liu2018attention}, $^\ddagger$: \citep{misra2016crossstitch}. Performance of other methods taken from \citep{liu2018attention}.
     }
    \label{tab:nyu_results_full}
\end{table*}

%

\end{document}
